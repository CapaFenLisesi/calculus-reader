\documentclass[../book/calcnotes.tex]{subfiles}

\begin{document}
\section{Arc length and surface area}
\label{sec:integral.arc-length-and-surface-area}
In this section, we'll consider how integration can help us solve two classes of geometric problem.

\subsection{Arc length}
\label{sec:integral.arc-length}
Consider the graph of $y = x^{3/2}$ over the interval $\closedint{0, 10}$, shown in \cref{fig:arclength:excurve}

\begin{smallfig}
  \begin{asy}
    smallsize();

    real f(real x) {return x^(3/2)/3;}

    xaxis("$x$", -.1, 11, Arrow);

    yaxis("$y$", 0, Arrow);

    funcplot(f, 0, 10);
  \end{asy}
  \caption{Graph of $y = x^{3/2}$ over $\closedint{0, 110}$}
  \label{fig:arclength:excurve}
\end{smallfig}

We have developed a variety of tools for studying the geometric properties of this curve; in particular, we know from \cref{sec:deriv.properties} how to compute the slopes of tangent lines to the curve and from \cref{sec:ftc} how to find the area under the curve.
However, we still don't know how to answer one seemingly simple question: \emph{how long is the curve?}

Let's physicalize the problem.
Suppose that the graph is the track of a roller-coaster, with height on the $y$-axis and horizontal distance on the $x$-axis.
Suppose that the cart is pulled by a mechanism on the ground, which moves to the right at a constant rate---say, \SI{1}{\meter\per\second}.
How fast does the cart move at a given point?

\emph{It depends on the slope of the tangent to the curve.}
At $x = 0$, for example, the cart is moving horizontally, so its velocity is exactly \SI{1}{\meter\per\second}.
However, at $x = \sfrac{1}{4}$, the cart is moving up, so its speed along the track must be considerably greater than \SI{1}{\meter\per\second} in order to keep up with the mechanism on the ground.
Exactly how fast is it moving there?

At the instant that $x = 4$, the cart's velocity points along the tangent line to the curve, which has slope $\eval*{\D{}{x} x^{3/2}}_{x = 4} = \sfrac{3}{2} \cdot \pbrac*{4}^{1/2} = 3$; to keep up with the horizontal motion, then, its velocity in the vertical axis must be \SI{3}{\meter\per\second}.
As illustrated in \cref{fig:arclength:veltri}, the total velocity $v$ of the cart satisfies $v^{2} = \pbrac*{\SI{1}{\meter\per\second}}^{2} + \pbrac*{\SI{3}{\meter\per\second}}^{2}$.
In other words, since $v$ is clearly positive, $v = \sqrt{\pbrac*{\SI{1}{\meter\per\second}}^{2} + \pbrac*{\SI{3}{\meter\per\second}}}^{2} = \sqrt{10} \si{\meter\per\second}$.

\begin{smallfig}
  \begin{asy}
    smallsize();

    import geometry;
    import math;

    pair a = (0, 0);
    pair b = (1, 0);
    pair c = (1, 1);

    draw(a--b, EndArrow);
    draw(a--c, EndArrow);
    draw(b--c, EndArrow);

    label("$v$", a--c, LeftSide);
    label("\SI{3}{\meter\per\second}", b--c, RightSide);
    label("\SI{1}{\meter\per\second}", a--b, RightSide);

    markrightangle(a, b, c);
    markangle("$\theta$", b, a, c);
  \end{asy}
  \caption{Cart velocity triangle at $x = 4$}
  \label{fig:arclength:veltri}
\end{smallfig}

Of course, nothing about this argument depended on the choice of $x = 4$.
At any value of $x$ along the curve, the velocity of the cart is given by $v(x) = \sqrt{1 + \pbrac*{\sfrac{3}{2} \sqrt{x}}^{2}}$.
We can then obtain the total distance traveled by integrating $v$ over the interval $\closedint{0, 10}$; thus, the arc length $L$ is given by
\begin{multline*}
  L
  = \Int{\sqrt{1 + \pbrac*{\frac{3}{2} \sqrt{x}}^{2}}}{x, 0, 10}
  = \Int{\sqrt{1 + \frac{9}{4} x}}{x, 0, 10}
  \\ = \underbrace{\eval*{\frac{8}{27} \pbrac*{1 + \frac{9}{4} x}^{3/2}}_{x = 0}^{x = 10}}_{\text{by substitution}}
  = \frac{94 \sqrt{94}}{27} - \frac{8}{27} \approx 33.458.
\end{multline*}

Moreover, nothing about this argument depended on the choice of the curve $y = x^{3/2}$ either!
We can use the same approach to find the arc length of any other curve.

\begin{theorem}[Arc length formula]
  \label{thm:arclength}
  Let $f$ be continuously differentiable over $\closedint{a, b}$.
  Then the length $L$ of the curve $y = f(x)$ over $x \in \closedint{a, b}$ is given by
  \begin{equation}
    \label{eq:arclength}
    L = \Int{\sqrt{1 + \pbrac*{f'(x)}^{2}}}{x, a, b}.
  \end{equation}
\end{theorem}

Unfortunately, integrals of the form appearing in \cref{eq:arclength} are notoriously difficult to solve exactly, so we're going to have a rough time of it if we want to do much more than the most elementary examples.
(See \cref{gp:arclength.catenary,gp:arclength.circle} for two important examples of cases where we \emph{can} compute the exact integral.)
However, \cref{thm:arclength} is still valuable for two reasons.
First, it gives us the theoretical machinery to solve an old and important problem, even if the solution turns out to be hard to work with in practice.
Second, combining it with the Riemann sum techniques in \cref{sec:riemann-sums} gives us powerful techniques for \emph{approximating} arc lengths; even if this isn't always philosophically satisfying, it's often extremely useful in practice.

\subsection{Surface area}
\label{sec:integral.surface-area}

\begin{gps}
  \begin{gp}
    \label{gp:arclength.circle}
    The unit circle is the set of points $\pbrac{x, y}$ in the Cartesian plane satisfying the equation $x^{2} + y^{2} = 1$.
    Using just this fact, we can calculate its perimeter!

    \begin{gpsol}
      To simplify things, let's work with just one-twelfth of the circle, from the point $\pbrac{0, 1}$ directly above the origin through an angle of $\sfrac{\pi}{6}$.
      Crucially, the point $\sfrac{\pi}{6}$ around the circle from $\pbrac{0, 1}$ has $x$-coordinate $\sfrac{1}{2}$, as shown in \cref{fig:arclength.circle.slice}:

      \begin{smallfig}
        \begin{asy}
          smallsize();

          import geometry;
          import math;

          xaxis("$x$", -.2, 1.2, Arrow);
          yaxis("$y$", -.2, 1.2, Arrow);

          pair O = (0, 0);

          real theta = 30;

          pair T = (0, 1);
          pair B = (Sin(theta), 0);
          pair P = (Sin(theta), Cos(theta));

          draw(T--O--P);
          draw(B--P, dotted);

          dot(B);
          label("$\sfrac{1}{2}$", B, NW);

          markangle("$\sfrac{\pi}{6}$", P, O, T);

          draw(Circle((0,0), 1), dashed);
          draw(arc(O, 1, 90-theta, 90));
          limits((-.1,-.1), (1.2,2), Crop);
        \end{asy}
        \caption{The unit circle with a $\sfrac{\pi}{6}$ slice}
        \label{fig:arclength.circle.slice}
      \end{smallfig}

      Along this arc, the points satisfy $y = \sqrt{1 - x^{2}}$, with derivative $y' = \frac{-x}{\sqrt{1 - x^{2}}}$.
      We can use \cref{thm:arclength} to compute the length of this arc, then multiply by $12$ to obtain the total perimeter $P$ of the circle
      \begin{multline*}
        P
        = 12 \Int{\sqrt{1 + \pbrac*{y'}^{2}}}{x, 0, \sfrac{1}{2}}
        = 12 \Int{\sqrt{1 + \pbrac*{\frac{-x}{\sqrt{1 - x^{2}}}}^{2}}}{x, 0, \sfrac{1}{2}} \\
        = 12 \Int{\sqrt{1 + \frac{x^{2}}{1 - x^{2}}}}{x, 0, \sfrac{1}{2}}
        = 12 \Int{\sqrt{\frac{1 - x^{2}}{1 - x^{2}} + \frac{x^{2}}{1 - x^{2}}}}{x, 0, \sfrac{1}{2}}
        = 12 \Int{\sqrt{\frac{1}{1 - x^{2}}}}{x, 0, \sfrac{1}{2}} \\
        = 12 \Int{\frac{1}{\sqrt{1 - x^{2}}}}{x, 0, \sfrac{1}{2}}
        = \eval*{12 \arcsin x}_{x = 0}^{x = \sfrac{1}{2}}
        = 12 \cdot \frac{\pi}{6} - 12 \cdot 0 = 2 \pi.
      \end{multline*}
      Good news!
    \end{gpsol}
  \end{gp}

  \begin{gp}
    \label{gp:arclength.catenary}
    A rope or chain hanging under its own weight will follow a \emph{catenary curve}, such as the one given by the equation
    \begin{equation*}
      y = \frac{e^{x} + e^{-x}}{2}.
    \end{equation*}
    (This function is sometimes called the \enquote{hyperbolic cosine}, for a variety of geometric and algebraic reasons.)
    What is the length of this curve over the interval $\closedint{-1, 1}$?

    \begin{gpsol}
      We can calculate directly that
      \begin{equation*}
        y' = \frac{e^{x} - e^{-x}}{2}.
      \end{equation*}
      Using \cref{thm:arclength}, we can then compute that the desired length is given by
      \begin{multline*}
        L
        = \Int{\sqrt{1 + \pbrac*{y'}^{2}}}{x, -1, 1}
        = \Int{\sqrt{1 + \pbrac*{\frac{e^{x} - e^{-x}}{2}}^{2}}}{x, -1, 1} \\
        = \Int{\sqrt{1 + \frac{e^{2x} - 2 + e^{-2x}}{4}}}{x, -1, 1}
        = \Int{\sqrt{\frac{e^{2x} + 2 + e^{-2x}}{4}}}{x, -1, 1} \\
        = \Int{\sqrt{\pbrac*{\frac{e^{x} + e^{-x}}{2}}^{2}}}{x, -1, 1}
        = \Int{\frac{e^{x} + e^{-x}}{2}}{x, -1, 1} \\
        = \eval*{\frac{e^{x} - e^{-x}}{2}}_{x = -1}^{x = 1}
        = \frac{e - e^{-1}}{2} - \frac{e^{-1} - e}{2}
        = e - \frac{1}{e}.
      \end{multline*}
      Whew!
    \end{gpsol}
  \end{gp}
\end{gps}

\begin{exercises}

\end{exercises}
\end{document}

%%% Local Variables:
%%% mode: latex
%%% TeX-master: "../book/calcnotes.tex"
%%% End: