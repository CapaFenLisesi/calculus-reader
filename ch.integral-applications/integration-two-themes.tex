\documentclass[../book/calcnotes.tex]{subfiles}

\begin{document}
\section{Two themes of integral applications}
\label{sec:integral.two-themes}

In \cref{sec:integral-theory}, we invested a \emph{lot} of energy into a pretty theoretical project: the development of a full theory of integration.
Now that we've done all this background work, we can begin to see the payoff; it turns out that integration has an array of applications at least as diverse as that of differentiation (which we explored in \cref{sec:derivative-applications}.
It turns out that most of the applications we'll see evoke one of two major themes, which we'll discuss a bit here before we dive into the details.

\subsection{Net change}
\label{sec:integral.net-change}
The second Fundamental Theorem of Calculus (\cref{thm:ftc.calc}) suggests an important interpretation of the signed area $\Int{f(x)}{x, a, b}$.
It doesn't just measure the area of a region in the plane; it measures the \defterm{accumulation} or \defterm{net change} of any quantity whose rate of change is given by $f$.
This interpretation is quite natural in settings where rates of change have physical interpretations.

\begin{example}
  \label{ex:flowint}
  Water is draining out of a tank through a pipe; at time $t$ (in \si{\minute}), its flow rate is $r(t)$ (in \si{\gallon\per\minute}).
  The definite integral $\Int{r(t)}{t,a,b}$ measures the total volume of water (in \si{\gallon}) that flows through the pipe during the time interval $t \in \closedint{a, b}$.
\end{example}

\begin{example}
  \label{ex:velocityint}
  A car is traveling on a highway from Minneapolis to Chicago; at time $t$ (in \si{\hour}), its velocity (in \si{\mile\per\hour}) is given by $v(t)$.
  The definite integral $\Int{f(t)}{t,a,b}$ measures the \defterm{total displacement} of the car (in \si{\mile}) over the time interval $t \in \closedint{a,b}$; in particular, if $v$ is nonnegative, then the integral measures the distance traveled by the car, which is the distance between the car's positions at time $a$ and time $b$.
  (If the car changes direction during its trip, this is not the same as total distance traveled; it is important to bear in mind the significance of sign in this calculation!)
\end{example}

Many of our applications of integration will take this form; if we want to find the net change in some quantity $A$ whose rate of change is given by $f(x) = \D{}{x} A(x)$, we can use a definite integral to do it.

\subsection{Product formulas}
\label{sec:integral.product-formulas}

Many familiar physical quantities are measured using \enquote{product formulas}.
To list just a few examples:
\begin{itemize}
\item
  The area $A$ of a rectangle of height $h$ and length $l$ is given by
  \begin{equation*}
    A = h l.
  \end{equation*}

\item
  The distance $D$ traveled by an object with velocity $V$ over time $T$ is given by
  \begin{equation*}
    D = V T.
  \end{equation*}

\item
  The work $W$ done by a force $F$ over a distance $D$ is given by
  \begin{equation*}
    W = F D.
  \end{equation*}

\item
  The volume $V$ of a cylinder with cross-sectional area $A$ and height $h$ is given by
  \begin{equation*}
    V = A h.
  \end{equation*}
\end{itemize}

% TODO: Add more!

Each of these formulas is extremely useful in its appropriate setting, but all are limited by the same fundamental constraint: they assume that the quantities involved are constant!
In practice, we rarely travel at a constant velocity, apply a constant force, or discover a perfect rectangle or cylinder in the real world.
Instead, we find situations where one of these quantities is variable and depends on something else---frequently position or time, although we'll see other independent variables as well.

We designed the definite integral to solve one specific problem: computing the area of a region whose height varies as a function of position.
It turns out, however, that it serves equally well in any of these other cases.
Each of the formulas above can be translated into an integral formula!
To wit:
\begin{itemize}
\item
  The area $A$ of the region under the curve $y = f(x)$ over the interval $\closedint{a, b}$ is given by
  \begin{equation*}
    \Int{f(x)}{x, a, b}.
  \end{equation*}

\item
  The distance $D$ traveled by an object with velocity $v(t)$ at time $t$ over the interval $\closedint{a, b}$ is given by
  \begin{equation*}
    D = \Int{v(t)}{t, a, b}.
  \end{equation*}

\item
  The work $W$ done by a force $f(x)$ at position $x$ over the interval $\closedint{a, b}$ is given by
  \begin{equation*}
    \Int{f(x)}{x, a, b}.
  \end{equation*}

\item
  The volume $V$ of an object with cross-sectional area $A(x)$ at position $x$ over the interval $\closedint{a, b}$ is given by
  \begin{equation*}
    V = \Int{A(x)}{x, a, b}.
  \end{equation*}
\end{itemize}

We'll see more of these kinds of applications as we move through the coming sections.

\begin{exercises}
\end{exercises}
\end{document}

%%% Local Variables:
%%% mode: latex
%%% TeX-master: "../book/calcnotes.tex"
%%% End:
