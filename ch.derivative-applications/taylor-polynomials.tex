\documentclass[../book/calcnotes.tex]{subfiles}

\begin{document}
\section{Taylor polynomials}
\label{sec:taylor-polynomials}

Consider the function $f(x) = \sin x + \cos x$.
What is $f(0.1)$?
Of course, we can use an electronic calculator to compute the approximation $f(0.1) \approx 1.09483758192\dots$---but how does it determine this value?
Can we replicate this process without relying on a magic black box?

As a first step, we should note that $0.1$ is close to $0$, and $f(x)$ is \emph{much} better behaved at $x = 0$: indeed, $f(0) = 1$ exactly.
Since $0.1 \approx 0$, we should expect that $f(0.1) \approx f(0) = 1$ as well, and this is borne out by the calculator result above.
But how far off is this approximation, and why does it go wrong?

\begin{smallfig}
  \begin{asy}
    smallsize();

    real xstart=-1, xend=1;
    real ystart=-1.6, yend=1.6;

    real a=0.1;

    xaxis("$x$", xstart, xend, Ticks(new real[] {a}, ticklabel=OmitFormat(a)), Arrow);
    labelx("$0.1$", a);
    yaxis("$y$", ystart, yend, Arrow);

    real f(real x) {return sin(x) + cos(x);}

    path fplot = funcplot(f, xstart, xend);
    draw(fplot);

    real taylorfunc(real x) {return 1;}
    path taylorplot = graph(taylorfunc, xstart, xend);
    draw(taylorplot, dashed);
  \end{asy}
  \caption{Plot of $y = \sin x + \cos x$ near $x = 0$ with its \enquote{constant approximation} $T_{0}$}
  \label{fig:taylor.constapprox}
\end{smallfig}

Let's take a look at this situation graphically.
Consider the plot of $y = f(x)$ shown in \cref{fig:taylor.constapprox}.
Our approximation effectively replaces the actual curve $y = \sin x + \cos x$ with the constant function $y = \sin 0 + \cos 0 = 1$, shown as a dashed line in \cref{fig:taylor.constapprox}.
Obviously, this isn't a very good approximation!
Still, it's a place to start.
For reasons that will be more clear shortly, we'll call this constant approximation function $T_{0}$.
, so $T_{0} (x) = f(0) = 1$ for all $x$.

How can we make this approximation better?
Our previous attempt involved picking a linear function $T_{0}$ to approximate $f$, but clearly it wasn't a very good choice: the values of $f$ can change, but $T_{0}$ is constant!
If we allow our approximating function to be a linear function with a nonzero slope, things will improve substantially.

Which line should we pick?
Clearly the line needs to go through the point $\pbrac*{0, f \pbrac{0}}$ to be a useful approximation,
Of all the lines passing through this point, the one that most closely resembles\footnote{Of course, this can be made formal, but it requires some fairly heavy machinery.} the curve $y = f(x)$ is its \emph{tangent line}, as illustrated in \cref{fig:taylor.linearapprox} so we'll use that as our next approximation, which we'll call $T_{1}$.

\begin{smallfig}
  \begin{asy}
    smallsize();

    real xstart=-1, xend=1;
    real ystart=-1.6, yend=1.6;

    real a=0.1;

    xaxis("$x$", xstart, xend, Ticks(new real[] {a}, ticklabel=OmitFormat(a)), Arrow);
    labelx("$0.1$", a);
    yaxis("$y$", ystart, yend, Arrow);

    real f(real x) {return sin(x) + cos(x);}

    path fplot = funcplot(f, xstart, xend);
    draw(fplot);

    real taylorfunc(real x) {return 1+x;}
    path taylorplot = graph(taylorfunc, xstart, xend);
    draw(taylorplot, dashed);
  \end{asy}
  \caption{Plot of $y = \sin x + \cos x$ near $x = 0$ with its \enquote{linear approximation} $T_{1}$}
  \label{fig:taylor.linearapprox}
\end{smallfig}

What is the slope of the tangent line?
Why, it's the derivative $f'(0)$, of course!
Thus, we can write a formula for this \enquote{linear approximation}\footnote{The inclusion of the seemingly strange \enquote{$x - 0$} term instead of the simpler \enquote{$x$} will make it easier to generalize this work to cases where the point of interest isn't $x = 0$.}: $T_{1} = f(0) + f'(0) \pbrac*{x - 0} = 1 + x$.
We can then use this to estimate the value of $f(0.1)$: $f(0.1) \approx T_{1} (0.1) = 1 + 1 \cdot 0.1 = 1.1$.
This is quite an improvement over the constant approximation, and it gets us nearly three decimal places of agreement with the calculator result.

Can we do better?
Certainly!
However, this is the best we can do \emph{with a linear function}.
Upgrading from a constant approximation to a linear one allowed us to take account of the fact that our function's values change with the value of $x$; in graphical terms, it allows for our function to have \enquote{slope}.
To improve the approximation further, we need to allow for the fact that the function's rate of change may itself be changing, or, in graphical terms, that the plot $y = f(x)$ may \enquote{bend}.
The natural next step after linear functions is quadratics, so let's try them for our next kind of approximation.

How can we find a useful quadratic approximation of $f$?
Let's think a bit more systematically about what we've done in the previous steps.
The first approximation we constructed, $T_{0}$, was chosen to have the same \emph{value} as $f$ at $x = 0$.
The next, $T_{1}$, was chosen to have the same \emph{value} and \emph{first derivative}.
This suggests that the right quadratic choice will have the same \emph{value}, \emph{first derivative}, and \emph{second derivative} as $f$ at $x = 0$.
It's clear that this should be of the form $T_{2} = 1 + x + ax^{2}$ for some coefficient $a$; the trick is in finding that coefficient.

If we're going to try to make $T_{2}$ have the same second derivative as $f$ at $x = 0$, we should compute that number $f''(0)$.
It is straightforward to compute that $\D[2]{}{x} \pbrac{\sin x + \cos x} = -\cos x - \sin x$, so $f''(0) = -1$.
What value of $a$ should we set so that $T''_{2} (0) = -1$ also?
We can calculate that $T''_{2} (x) = 2a$, so evidently we need to set $a = -\sfrac{1}{2}$.
Thus, the best quadratic approximation is $T_{2} = f(0) + f'(0) \pbrac*{x - 0} + \frac{1}{2} f''(0) \pbrac*{x - 0}^{2} = 1 + x - \frac{1}{2} x^{2}$, as illustrated in \cref{fig:taylor.quadraticapprox}.

\begin{smallfig}
  \begin{asy}
    smallsize();

    real xstart=-.2, xend=3.2;
    real ystart=-1.6, yend=1.6;

    real a=0.1;

    xaxis("$x$", xstart, xend, Ticks(new real[] {a}, ticklabel=OmitFormat(a)), Arrow);
    labelx("$0.1$", a);
    yaxis("$y$", ystart, yend, Arrow);

    real f(real x) {return sin(x) + cos(x);}

    path fplot = funcplot(f, xstart, xend);
    draw(fplot);

    real taylorfunc(real x) {return 1+x-x^2/2;}
    path taylorplot = graph(taylorfunc, xstart, xend);
    draw(taylorplot, dashed);
  \end{asy}
  \caption{Plot of $y = \sin x + \cos x$ near $x = 0$ with its \enquote{quadratic approximation} $T_{2}$}
  \label{fig:taylor.quadraticapprox}
\end{smallfig}

What if we wanted to construct a higher-order polynomial approximation?
Exactly the same idea will apply: if we want an $n$th-degree polynomial approximation to $f$ near $x = 0$, we simply need to find the $n$th-degree polynomial whose first $n$ derivatives at $x = 0$ are equal to those of $f$.
To achieve this, we just need to set the coefficient on the $\pbrac{x - 0}^{i}$ term to the right value.
But what is that value?
The right coefficient of degree $1$ was just $f'(0)$, but the right coefficient of degree $2$ was $\frac{1}{2} f''(0)$, because taking the second derivative multiplies this coefficient by $2$.
If we keep going, the right coefficient of degree $3$ will turn out to be $\frac{1}{3 \cdot 2 \cdot 1} f'''(0)$, the coefficient of degree $4$ will be $\frac{1}{4 \cdot 3 \cdot 2 \cdot 1} f''''(0)$, and so on.
In general, the coefficient on $\pbrac{x - 0}^{i}$ will need to be $\frac{1}{i!} f^{(i)}(0)$ to get the derivatives to line up.

We should also note at this point that there's nothing special about using $x = 0$ as the \enquote{center point} for our approximation.
We could have used any value $x = a$ we wanted; then we'd need to take derivatives of $f$ at $a$ and use them as coefficients on terms of the form $\pbrac*{x - a}^{n}$.

This leads us to our complete definition of these approximating polynomials, usually known in the literature as \enquote{Taylor polynomials}:
\begin{definition}
  \label{def:taylorpoly}
  Let $f$ be $n$ times differentiable at $x = a$.
  Then the \defterm{$n$th Taylor polynomial of $f$ centered at $x = a$}[Taylor polynomial] is
  \begin{equation}
    \label{eq:taylorpoly}
    T_{n} (x) = f(a) + f'(a) \pbrac{x - a} + \frac{1}{2} f''(a) \pbrac{x - a}^{2} + \dots + \frac{1}{n!} f^{(n)} (a) \pbrac*{x - a}^{n}.
  \end{equation}
  More compactly, and using the conventions that $f^{(0)} (x) = f(x)$ and $0! = 1$, we can write
  \begin{equation}
    \label{eq:taylorpoly.sigma}
    T_{n} (x) = \sum_{i = 0}^{n} \frac{f^{(i)} (a)}{i!} \pbrac*{x - a}^{i}.
  \end{equation}
  The $n$th Taylor polynomial centered at $x = 0$ is sometimes called the \defterm{Maclaurin polynomial}.
\end{definition}

Continuing with our example from before, it is straightforward to compute the $n$th Taylor polynomial to $f(x) = \sin x + \cos x$ centered at $x = 0$ for any $n$ we choose.
Each of these polynomials is a better approximation to $f(x)$ than the one before it, as we will see.

\begin{example}
  Find the $n$th Taylor polynomial to $f(x) = \sin x + \cos x$ centered at $x = 0$ for $n \in \cbrac{1, 2, 3, 4, 5}$.
  Use each to approximate $f(0.1)$.
\end{example}

\begin{soln}
  We're going to need to know the values of the first five derivatives of $f$ at $x = 0$.
  Fortunately, these are easy to compute, since $\sin x$ and $\cos x$ both have well-behaved derivatives.
  The results of this calculation are shown in \cref{tab:taylor.derivatives}.

  \begin{table}[htbp]
    \centering
    \begin{tabular}{c c c c}
      \toprule
      $n$ & $\eval*{\D[n]{}{x} \sin x}_{x = 0}$ & $\eval*{\D[n]{}{x} \cos x}_{x = 0}$ & $\eval*{\D[n]{}{x} f(x)}_{x = 0}$ \\
      \midrule
      $0$ & $0$ & $1$ & $1$ \\
      $1$ & $1$ & $0$ & $1$ \\
      $2$ & $0$ & $-1$ & $-1$ \\
      $3$ & $-1$ & $0$ & $-1$ \\
      $4$ & $0$ & $1$ & $1$ \\
      $5$ & $1$ & $0$ & $1$ \\
      \bottomrule
    \end{tabular}
    \caption{The first five derivatives of $f(x) = \sin x + \cos x$ at $x = 0$}
    \label{tab:taylor.derivatives}
  \end{table}

  With these values of $f^{(i)} (0)$ in hand, we can use \cref{def:taylorpoly} to find the desired Taylor polynomials.
  Expanding \cref{eq:taylorpoly}, we find that
  \begin{align*}
    T_{0} (x) &= \frac{1}{0!}. \\
    T_{1} (x) &= \frac{1}{0!} + \frac{1}{1!} x. \\
    T_{2} (x) &= \frac{1}{0!} + \frac{1}{1!} x + \frac{-1}{2!} x^{2}. \\
    T_{3} (x) &= \frac{1}{0!} + \frac{1}{1!} x + \frac{-1}{2!} x^{2} + \frac{-1}{3!} x^{3}. \\
    T_{4} (x) &= \frac{1}{0!} + \frac{1}{1!} x + \frac{-1}{2!} x^{2} + \frac{-1}{3!} x^{3} + \frac{1}{4!} x^{4}. \\
    T_{5} (x) &= \frac{1}{0!} + \frac{1}{1!} x + \frac{-1}{2!} x^{2} + \frac{-1}{3!} x^{3} + \frac{1}{4!} x^{4} + \frac{1}{5!} x^{5}.
  \end{align*}

  The plot of $f$ and the higher Taylor polynomials $T_{3}$, $T_{4}$ and $T_{5}$ is shown in \cref{fig:taylor.manyapprox}.
  Notice how each successive Taylor polynomial \enquote{looks like} $f$ on a wider $x$-interval.
  \begin{medfig}
    \begin{asy}
      medsize();

      real xmin=-12, xmax=12;
      real ymin=-6, ymax=6;

      xaxis("$x$", xmin, xmax, Arrow);
      yaxis("$y$", ymin, ymax, Arrow);

      real f(real x) {return sin(x) + cos(x);}

      path fplot = funcplot(f, xmin, xmax);

      real taylor3(real x) {return 1 + x - x^2/2 - x^3/6;}
      path plot3 = funcplot(taylor3, xmin, xmax, dash1);

      real taylor4(real x) {return 1 + x - x^2/2 - x^3/6 + x^4/24;}
      path plot4 = funcplot(taylor4, xmin, xmax, dash2);

      real taylor5(real x) {return 1 + x - x^2/2 - x^3/6 + x^4/24 + x^5/120;}
      path plot5 = funcplot(taylor5, xmin, xmax, dash3);

      limits((xmin,ymin),(xmax,ymax),Crop);
    \end{asy}
    \caption{Plot of $y = \sin x + \cos x$ near $x = 0$ with its third, fourth, and fifth Taylor polynomials}
    \label{fig:taylor.manyapprox}
  \end{medfig}

  Now that we have the desired Taylor polynomials $T_{i} (x)$, we can use them to estimate $f(0.1)$ simply by evaluating $T_{i} (0.1)$ for each $i$:
  \begin{align*}
    T_{0} \pbrac{0.1} &= \frac{1}{0!} = 1. \\
    T_{1} \pbrac{0.1} &= \frac{1}{0!} + \frac{1}{1!} \pbrac{0.1} = \frac{11}{10} = 1.1 \\
    T_{2} \pbrac{0.1} &= \frac{1}{0!} + \frac{1}{1!} \pbrac{0.1} + \frac{-1}{2!} \pbrac{0.1}^{2} = \frac{219}{200} = 1.095. \\
    T_{3} \pbrac{0.1} &= \frac{1}{0!} + \frac{1}{1!} \pbrac{0.1} + \frac{-1}{2!} \pbrac{0.1}^{2} + \frac{-1}{3!} \pbrac{0.1}^{3} = \frac{6569}{6000} = 1.0948\overline{33}. \\
    T_{4} \pbrac{0.1} &= \frac{1}{0!} + \frac{1}{1!} \pbrac{0.1} + \frac{-1}{2!} \pbrac{0.1}^{2} + \frac{-1}{3!} \pbrac{0.1}^{3} + \frac{1}{4!} \pbrac{0.1}^{4} \\ &= \frac{87587}{80000} = 1.0948375. \\
    T_{5} \pbrac{0.1} &= \frac{1}{0!} + \frac{1}{1!} \pbrac{0.1} + \frac{-1}{2!} \pbrac{0.1}^{2} + \frac{-1}{3!} \pbrac{0.1}^{3} + \frac{1}{4!} \pbrac{0.1}^{4} + \frac{1}{5!} \pbrac{0.1}^{5} \\ &= \frac{13138051}{12000000} \approx 1.09483758\overline{33}.
  \end{align*}
\end{soln}

Comparing the value $T_{5} \pbrac{0.1}$ to the calculator value of $\sin 0.1 + \cos 0.1$ from the start of the section, we find nine decimal places of accuracy---not bad for a calculation we can write down on a napkin!
In fact, this is exactly how some calculators compute values of functions like $\sin$, $\cos$, $\exp$, and $\ln$.
It's a very powerful computational technique, because it lets us replace messy \enquote{transcendental functions} with the simple arithmetic operations of multiplication and addition.

% TODO: Add material about error bounds?

\begin{gps}
  \begin{gp}
    Let $f(x) = \sqrt{x}$.
    Find the linear approximation to $f$ centered at $x = 64$.
    Use it to estimate the value of $\sqrt{63}$.
  \end{gp}

  \begin{gp}
    Use the first five Taylor polynomials to $f(x) = e^{x}$ centered at $x = 0$ to approximate the value of $e$.
    Compare the results to the decimal value of $e$ given by a calculator or Web site.
  \end{gp}

  \begin{gp}
    Find the first few Taylor polynomials for $f(x) = \frac{1}{1 - x}$ and graph them along with $y = f(x)$.
    For what values of $x$ do these polynomials approximate $f(x)$?
  \end{gp}
\end{gps}

\begin{exercises}
\end{exercises}
\end{document}

%%% Local Variables:
%%% mode: latex
%%% TeX-master: "../book/calcnotes.tex"
%%% End: