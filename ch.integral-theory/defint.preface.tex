\documentclass[../book/calcnotes.tex]{subfiles}

\begin{document}
\section*{Preface}
\label{sec:defint.preface}
We now turn our attention to a brand new problem.

\begin{motprob}
  Given a continuous function $f$ which is non-negative over an interval $I = \closedint{a, b}$, calculate the area of the region above the $x$-axis and below the curve $y = f(x)$ over the interval $I$.
\end{motprob}

\begin{motex}
  Consider the function $f$ whose graph is shown in \cref{fig:boundregex}.
  What is the area of the shaded region?

  \begin{marginfigure}
    \centering
    \begin{tikzpicture}
      \begin{axis}[
        agdplot,
        rectify,
        ymin = -.4,
        ymax = 3,
        ytick = \empty,
        xtick = {1,3},
        ]
        \addplot+ [domain=1:3] {1.5+sin(deg(x^2))} \closedcycle;

        \addplot [thick, domain=0:3.5] {1.5+sin(deg(x^2))} node [pos=0.65,pin={130:$y = f(x)$}] {};

      \end{axis}
    \end{tikzpicture}
    \caption{Region bounded by $y = f(x)$ over $\closedint{1, 3}$}
    \label{fig:boundregex}
  \end{marginfigure}
\end{motex}

This kind of problem has a long history, and many of the greatest mathematical thinkers of the last millenium have dedicated substantial energy to it.
Fortunately, it won't take \emph{us} a thousand years to solve!
Still, we do have our work cut out for us.
\end{document}

%%% Local Variables:
%%% mode: latex
%%% TeX-master: "../book/calcnotes.tex"
%%% End: