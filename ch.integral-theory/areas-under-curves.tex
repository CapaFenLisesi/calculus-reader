\documentclass[../book/calcnotes.tex]{subfiles}

\begin{document}
\section{Areas under curves}
\label{sec:areasundercurves}

As usual, we'll start by laying the groundwork with some foundational definitons and properties.

\subsection{Basic definitions}
\label{sec:areas.defs}
We're going to be talking about these areas under curves quite a bit, so we should start by giving them a name.

\begin{definition}
  \label{def:defint.pos}
  Let $f$ be a function which is continuous and non-negative over an interval $\closedint{a, b}$.
  The \emph{definite integral} of $f$ over $\closedint{a,b}$, denoted $\Int{f(x)}{x,a,b}$, is the area of the region in the plane bounded by the curves $y = 0$, $x = a$, $x = b$, and $y = f(x)$.
  \marginpar{\textbf{Caution!} \Cref{def:defint.pos} is not our permanent definition. Find that below in \cref{def:defint}.}
\end{definition}

\begin{note*}
  We didn't mention the variable $x$ in the definition of $\Int{f(x)}{x,a,b}$!
  This is because the variable $x$ is a \enquote{dummy variable}.
  We could just as easily have called the independent variable $\xi$; then $\Int{f(\xi)}{\xi,a,b}$ would represent \emph{exactly the same area} as $\Int{f(x)}{x,a,b}$.
\end{note*}

Sometimes, the definite integral of a function is straightforward to compute.
For example, the region in question may be a nice geometric shape.

\begin{example}
  \label{ex:inttrap}
  Compute $\Int{2x}{x,1,3}$.
\end{example}

\begin{soln}
  As shown in \cref{fig:int2x}, the region bounded by $y = 2x$, $y = 1$, $x = 0$, and $x = 3$ is a trapezoid with base $2$ and heights $2$ and $6$.
  Therefore, $\Int{2x}{x,1,3} = 2 \cdot \frac{2 + 6}{2} = 8$.

  \begin{smallfig}
    \centering
    \begin{asy}
      smallsize();

      real f(real x) {return x/2;}

      real start=0, end=3.5;

      real a=1, b=3;

      xaxis("$x$",start,end,Ticks(new real[] {a,b}), Arrow);
      yaxis("$y$",-.2, Arrow);

      path fplot = funcplot(f, start, end);
      draw(fplot);

      fillunder(f, a, b, box1);
    \end{asy}
    \caption{Region bounded by $y = 2x$ over $\closedint{1,3}$}
    \label{fig:int2x}
  \end{smallfig}
\end{soln}

\begin{example}
  \label{ex:intcirc}
  Compute $\Int{\sqrt{9 - x^{2}}}{x,-3,3}$.
\end{example}

\begin{soln}
  As shown in \cref{fig:intcirc}, the region bounded by $y = \sqrt{9 - x^{2}}$, $y = 0$, $x = -3$, and $x = 3$ is a semicircle of radius $3$.
  Therefore, $\Int{\sqrt{9 - x^{2}}}{x,-3,3} = \frac{1}{2} \pi \pbrac*{3}^{2} = \frac{9}{2} \pi$.

  \begin{smallfig}
    \centering
    \begin{asy}
      smallsize();

      real start=-4, end=4;

      real a=-3, b=3;

      real center = (b+a)/2;
      real radius = (b-a)/2;

      xaxis("$x$",start,end,Ticks(new real[] {a,b}), Arrow);
      yaxis("$y$",-.2,end,Arrow);

      path fplot = arc((center, 0), radius, 0, 180);
      draw(fplot);

      fill(fplot -- (a, 0) -- (b, 0) -- cycle, box1);
    \end{asy}
    \caption{Region bounded by $y = \sqrt{9-x^{2}}$ over $\closedint{-3,3}$}
    \label{fig:intcirc}
  \end{smallfig}
\end{soln}

Of course, if we were only interested in finding areas of triangles and circles, we wouldn't need to bring calculus into it!
The problems start when the regions \emph{don't} make nice geometric shapes (as was the case in \cref{fig:boundregex}).
Still, this is a good place to start; for \enquote{nice} functions, we can just fall back on the geometry you learned years ago.

\subsection{Properties of definite integrals}
\label{sec:defint.properties}
In \cref{def:defint.pos}, we defined the definite integral $\Int{f(x)}{x,a,b}$ of a function---under the condition that the function is non-negative on $\closedint{a,b}$.
To develop an algebraic theory of integrals, we'll need to do better than that.
But what would it even mean to take an integral of a function whose values are negative?
We haven't defined that yet!
Of course, we're free to take any definition we'd like, but, as usual, some definitions are \emph{better} than others.

Let's step back from that question a moment and think about the algebraic properties of definite integrals.
Once we identify the properties we want our definition to have, it will be much easier to say what that definition should be!

\subsubsection*{Constant multiples}
Suppose that, for a certain function $f$, we know the define integral $\Int{f(x)}{x,a,b}$.
Is this enough information to compute $\Int{k f(x)}{x,a,b}$ for some given constant $k \in \RR$?

\begin{example}
  \label{ex:int.constmul}
  Let $f(x) = 2x$ and let $k = 2$.
  We showed in \cref{ex:inttrap} that $\Int{f(x)}{x,1,3} = 8$.
  By similar arguments, $\Int{k f(x)}{x,1,3} = \Int{4x}{x,1,3} = 16$.
  But this is equal to twice the original integral---in other words, $k \Int{f(x)}{x,1,3} = \Int{k f(x)}{x,1,3}$ in this situation.
\end{example}

Will this always work?
In fact, it will!
Multiplying $f$ by $k$ simply \enquote{stretches} the graph of $y = f(x)$ vertically by a factor of $k$; a piece of the plane with area $1$ before this operation will have area $k$ afterwards.

Of course, this argument only makes sense if $f$ is a non-negative function over $\closedint{a,b}$ and $k \geq 0$; otherwise, $k f(x)$ won't be compatible with \cref{def:defint.pos}.
But this gives us a clue about how to handle the problem of negative-valued functions!

However we end up defining the integral of an arbitrary function, we'd like to keep this delightful propertly that $\Int{k f(x)}{x,a,b} = k \Int{f (x)}{x,a,b}$.
Suppose $f$ is a non-positive function and that we know the value of $\Int{f(x)}{x,a,b}$; if our property is to hold, we'd better have that $\Int{f(x)}{x,a,b} = -\Int{-f(x)}{x,a,b}$.
But this implies that $\Int{f(x)}{x,a,b}$ is a negative number!
It's still equal in absolute value to the area bounded between $y = 0$ and $y = f(x)$ over $\closedint{a,b}$, but the sign has been flipped, since $y = f(x)$ is now \emph{below} $y = 0$.
This turns out to be enough to build our general definition of the definite integral:

\begin{definition}
  \label{def:defint}
  Let $f$ be a function which is continuous over an interval $\closedint{a,b}$.
  The \emph{definite integral} of $f$ over $\sbrac{a, b}$, denoted $\Int{f(x)}{x,a,b}$, is the \enquote{signed area} of the region bounded by the curves $y = 0$, $y = f(x)$, $x = a$, and $x = b$, computed by measuring the area of the regions where $y=f(x)$ is below $y = 0$ and subtracting it from the area of the regions where $y=f(x)$ is above $y = 0$.
  If $\Int{f(x)}{x,a,b}$ is defined, we say that $f$ is \emph{integrable} over $\closedint{a,b}$.
\end{definition}

\begin{example}
  Compute $\Int{3x}{x,-1,2}$.
\end{example}

\begin{soln}
  We need to consider the region bounded by $y = 3x$, $y = 0$, $x = -1$, and $x = 2$.
  As shown in \cref{fig:intpm}, this region has both \enquote{positive} and \enquote{negative} pieces.
  The \enquote{negative} piece is a triangle of sides $1$ and $3$, so its area is $\frac{1}{2} \cdot 1 \cdot 3 = \frac{3}{2}$.
  Similarly, the \enquote{positive} piece is a triangle with sides $2$ and $6$, so its area is $\frac{1}{2} \cdot 2 \cdot 6 = 6$.
  Therefore, $\Int{3x}{x,-1,2} = 6 - \frac{3}{2} = \frac{9}{2}$.

  \begin{smallfig}
    \centering
    \begin{asy}
      smallsize();

      real f(real x) {return x/2;}

      real start=-1.2, end=2.2;
      real a=-1, b=0, c=2;

      xaxis("$x$",start,end,LeftTicks(new real[] {a,c}), Arrow);
      yaxis("$y$",-.2,Arrow);

      path fplot = funcplot(f,start,end);
      draw(fplot);

      fillunder(f, a, b, box1);
      fillunder(f, b, c, box2);
    \end{asy}
    \caption{Region bounded by $y = 3x$ over $\closedint{-1,2}$}
    \label{fig:intpm}
  \end{smallfig}
\end{soln}

With this definition in hand, we can state this idea as a theorem.

\begin{theorem}[Constant multiple rule for integrals]
  \label{thm:int.constmult}
  For any function $f$ which is integrable over $\closedint{a,b}$ and any constant $k \in \RR$,
  \begin{equation}
    \label{eq:int.constmul}
    \Int{k f(x)}{x,a,b} = k \Int{f(x)}{x,a,b}.
  \end{equation}
\end{theorem}

\subsubsection*{Sums}
Now suppose that, for two non-negative functions $f$ and $g$, we know the values $\Int{f(x)}{x,a,b}$ and $\Int{g(x)}{x,a,b}$.
Is this enough information to compute $\Int{f(x) + g(x)}{x,a,b}$?

\begin{example}
  \label{ex:int.sum}
  Let $f(x) = 1$ and $g(x) = 2x$.
  Consider their integrals over $\closedint{1,3}$.
  We showed in \cref{ex:inttrap} that $\Int{g(x)}{x,1,3} = 8$.
  Since the region under $y = f(x)$ over $\closedint{1,3}$ is just a rectangle of height $1$ and base $2$, its area s $2$, so this is the value of $\Int{f(x)}{x,1,3}$.
  Now consider the function $h(x) = f(x) + g(x)$.
  What is $\Int{h(x)}{x,1,3}$?
  As shown in \cref{fig:intstack}, this is the area of a trapezoid with base $2$ and heights $3$ and $7$, so $\Int{h(x)}{x,1,3} = 2 \cdot \frac{3+7}{2} = 10$.
  But this is also $\Int{f(x)}{x,1,3} + \Int{g(x)}{x,1,3}$!

  \begin{marginfigure}
    \centering
    \begin{asy}
      smallsize();

      real f(real x) {return x/2;}
      real g(real x) {return x/2+1/4;}

      real start=0, end=3.5;

      real a=1, b=3;

      xaxis("$x$",start,end,Ticks(new real[] {a,b}), Arrow);
      yaxis("$y$",-.2, Arrow);

      path fplot = funcplot(f, start, end);
      draw(fplot);

      path ffill = graph(f,a,b,operator ..);
      fill(ffill -- (b, 0) -- (a, 0) -- cycle, box1);

      path gplot = graph(g,start,end,operator ..);
      draw(gplot);

      path gfill = graph(g,a,b,operator ..);
      fill(gfill -- (b, f(b)) -- (a, f(a)) -- cycle, box2);
    \end{asy}
    \caption{Region bounded by $y = 2x+1$ over $\closedint{1,3}$}
    \label{fig:intstack}
  \end{marginfigure}
\end{example}

Will this always work?
It turns out that it will---and the reason why is surprisingly straightforward.
Consider \cref{fig:intstack} again, which shows the region of the plane whose area is $\Int{f(x) + g(x)}{x, a, b}$.
The lower piece is exactly the region bounded by $y = g(x)$, so clearly its area is $\Int{g(x)}{x,a,b}$.

But what about the upper piece?
This was formed by taking the region bounded by $y = f(x)$ and \enquote{sliding it up} so it sits on top of the curve $y = g(x)$.
Crucially, the length of each vertical line in this piece is unchanged; they're merely skewed from a rectangle into a parallelogram.
It turns out\sidefootnote{This result is known as \enquote{Cavlieri's principle}, after fifteenth-century Italian mathematician Bonaventura Cavalieri, although it was known in various forms for hundreds of years before him.} that this means that the rectangle and the parallelogram have the same area!
(To see why, imagine taking a tall stack of index cards and setting it edges-down on an uneven surface; the individual cards will slide up and down to match the contour of the surface, but the volume of the stack is clearly unchanged.)

From this, we obtain another important theorem about definite integrals.

\begin{theorem}[Sum rule for integrals]
  \label{thm:int.sum}
  For any functions $f$ and $g$ which are integrable over $\closedint{a,b}$,
  \begin{equation}
    \label{eq:int.sum}
    \Int{f(x) + g(x)}{x,a,b} = \Int{f(x)}{x,a,b} + \Int{g(x)}{x,a,b}.
  \end{equation}
\end{theorem}

\subsubsection*{Combining intervals}

Now suppose that we know several different integrals of the \emph{same} function; for example, perhaps we know $\Int{f(x)}{x,a,b}$ and $\Int{f(x)}{x,b,c}$.
Is this enough information to compute $\Int{f(x)}{x,a,c}$?

\begin{example}
  \label{ex:int.split}
  Let $f(x) = \sqrt{9 - x^{2}}$.
  Find $\Int{f(x)}{x,-3,0}$ and $\Int{f(x)}{x,0,3}$, then compare to $\Int{f(x)}{x,-3,3}$ as computed in \cref{ex:intcirc}.
\end{example}

\begin{soln}
  Each of $\Int{f(x)}{x,-3,0}$ and $\Int{f(x)}{x,0,3}$ represents the area of one-quarter of a circle of radius $3$, as illustrated in \cref{fig:intcircsplit}.
  Therefore, each is equal to $\frac{1}{4} \pi \pbrac*{3}^{2} = \frac{9}{4} \pi$.
  These two values add up to $\frac{9}{2} \pi$, which was the value of $\Int{f(x)}{x,-3,3}$ computed in \cref{ex:intcirc}.

  \begin{smallfig}
    \centering
    \begin{asy}
      smallsize();

      real start=-4, end=4;

      real a=-3, b=3;

      real center = (b+a)/2;
      real radius = (b-a)/2;

      xaxis("$x$",start,end,Ticks(new real[] {a,b}), Arrow);
      yaxis("$y$",-.2,end,Arrow);

      path leftarc = arc((center, 0), radius, 90, 180);
      draw(leftarc);
      fill(leftarc -- (center, 0) -- cycle, box1);

      path rightarc = arc((center, 0), radius, 0, 90);
      draw(rightarc);
      fill(rightarc -- (center, 0) -- cycle, box2);
    \end{asy}
    \caption{Region bounded by $y = \sqrt{9-x^{2}}$ over $\closedint{-3,3}$}
    \label{fig:intcircsplit}
  \end{smallfig}
\end{soln}

Will this always work?
Once again, it turns out that it will, as is clearly seen from \cref{fig:intcircsplit}.
Fixing an integrable function $f$ and three values $a < b < c$, it is clear that $\Int{f(x)}{x,a,b}$ and $\Int{f(x)}{x,b,c}$ represent the (signed) areas of two regions which, when combined, form the region whose (signed) area is measured by $\Int{f(x)}{x,a,c}$.

Similarly, suppose that we want to compute an integral $\Int{f(x)}{x,b,a}$ where $a < b$.
Once again, we'd like to preserve the lovely property that $\Int{f(x)}{x,a,b} + \Int{f(x)}{x,b,c} = \Int{f(x)}{x,a,c}$.
We should consider two interesting special cases of this equation.
\begin{itemize}
\item
  If we set $a = b = c$, this implies that we should have $\Int{f(x)}{x,a,a} + \Int{f(x)}{x,a,a} = \Int{f(x)}{x,a,a}$.
  In other words, $2 \Int{f(x)}{x,a,a} = \Int{f(x)}{x,a,a}$.
  This can only be true if $\Int{f(x)}{x,a,a} = 0$.

\item
  Meanwhile, if we set $c = a$, this implies that we should have $\Int{f(x)}{x,a,b} + \Int{f(x)}{x,b,a} = \Int{f(x)}{x,a,a} = 0$.
  This can only be true if $\Int{f(x)}{x,a,b} = - \Int{f(x)}{x,b,a}$.
\end{itemize}

We've now collected three more more important results about definite integrals, which we will collect into a single theorem.

\begin{theorem}[Combined intervals rule for integrals]
  \label{thm:int.combint}
  Let $f$ be integrable over an interval $\closedint{a,b}$.
  Then
  \begin{equation}
    \label{eq:int.reverse}
    \Int{f(x)}{x,a,b} = - \Int{f(x)}{x,b,a}
  \end{equation}
  and
  \begin{equation}
    \label{eq:int.trivint}
    \Int{f(x)}{x,a,a} = 0.
  \end{equation}
  If $f$ is also integrable over the interval $\closedint{b,c}$, then
  \begin{equation}
    \label{eq:int.split}
    \Int{f(x)}{x,a,b} + \Int{f(x)}{x,b,c} = \Int{f(x)}{x,a,c}.
  \end{equation}
\end{theorem}

\begin{exercises}
\end{exercises}
\end{document}

%%% Local Variables:
%%% mode: latex
%%% TeX-master: "../book/calcnotes.tex"
%%% End: