\documentclass[../book/calcnotes.tex]{subfiles}

\begin{document}
\section{Integration by substitution}
\label{sec:int.subs}
In light of \cref{thm:ftc.deriv,thm:ftc.calc}, if we want to develop a theory of definite integrals, what we really need is a theory of antiderivatives.
We've already explored this topic in \cref{sec:antiderivatives}, but there's a lot more we can do!

As a first step, let's investigate what happens when we integrate composite functions.
\begin{example}
  \label{ex:integrate.composite}
  Let $f(x) = \sin x$.
  Compare the definite integrals $\Int{f(x)}{x, 0, \pi}$ and $\Int{f(2x)}{x, 0, \sfrac{\pi}{2}}$.
\end{example}

\begin{soln}
  As a first step, let's think about what each of these integrals represents.
  $\Int{f(x)}{x, 0, \pi}$ is the result of \enquote{adding up} all the values of $\sin x$ as $x$ ranges over $\closedint{0, \pi}$, while $\Int{f(2x)}{x, 0, \sfrac{\pi}{2}}$ is the result of \enquote{adding up} all the values of $\sin 2x$ as $x$ ranges over $\closedint{0, \frac{\pi}{2}}$.
  This seems to suggest that they might have the same value!

  However, there is a flaw in our reasoning.
  We note that $-\cos x$ is an antiderivative of $f(x)$ and $- \frac{1}{2} \cos 2x$ is an antiderivative of $f(2x)$.
  Thus, we can calculate the integrals explicitly; $\Int{f(x)}{x, 0, \pi} = \pbrac*{- \cos \pi} - \pbrac*{- \cos 0} = 2$, while $\Int{f(2x)}{x, 0, \sfrac{\pi}{2}} = \pbrac*{-\frac{1}{2} \cos 2 \frac{\pi}{2}} - \pbrac*{-\frac{1}{2} \cos 2 \cdot 0} = 1$.
  Thus, $\Int{f(x)}{x, 0, \pi} = 2 \Int{f(2x)}{x, 0, \sfrac{\pi}{2}}$.
\end{soln}

What went wrong with our initial reasoning there?
Let's take a look at the graphical situation, illustrated in \cref{fig:subsex}, to see if we can make sense of the problem.

\begin{medfig}
  \parbox{.6\textwidth}{
    \subbottom[\enquote{Wide} function\label{fig:subsex.wide}]{
      \begin{tikzpicture}
        \begin{axis}[
          agdplot,
          rectify,
          ymin = -.2,
          ymax = 1.2,
          ytick = \empty,
          xtick = {0,1.5708,3.1415},
          xticklabels = {$0$, $\sfrac{\pi}{2}$, $\pi$},
          ]
          \addplot+ [domain=0:pi,box1] {sin(deg(x))} \closedcycle;

          \addplot [thick, domain=-.2:pi+.2] {sin(deg(x))} node [pos=0.55,pin={0:$y = \sin x$}] {};
        \end{axis}
      \end{tikzpicture}
    }
  }
  \quad
  \parbox{.35\textwidth}{
    \subbottom[\enquote{Narrow} function\label{fig:subsex.narrow}]{
      \begin{tikzpicture}
        \begin{axis}[
          agdplot,
          rectify,
          ymin = -.2,
          ymax = 1.2,
          ytick = \empty,
          xtick = {0,0.7854,1.5708},
          xticklabels = {$0$, $\sfrac{\pi}{4}$, $\sfrac{\pi}{2}$},
          ]
          \addplot+ [domain=0:pi/2,box1] {sin(2*deg(x))} \closedcycle;

          \addplot [thick, domain=-.2:pi/2+.2] {sin(2*deg(x))} node [pos=0.4,pin={-85:$y = \sin 2x$}] {};
        \end{axis}
      \end{tikzpicture}
    }
  }
  \label{fig:subsex}
  \caption{Two related integrals}
\end{medfig}

Each of $\Int{\sin x}{x, 0, \pi}$ and $\Int{\sin 2x}{x, 0, \sfrac{\pi}{2}}$ represents an area, represented in \cref{fig:subsex.wide} and \cref{fig:subsex.narrow} respectively.
But these areas are clearly not equal!
Multiplying the variable $x$ by $2$ \enquote{compresses} the region in \cref{fig:subsex.narrow} by a factor of $2$ in the horizontal direction, just as multiplying the function by $\frac{1}{2}$ does in the vertical direction.

The key lesson here is that we cannot blithely substitute one function for another when computing integrals.
We should think more carefully about what happens in these situations.

Thinking back to \cref{sec:chainrule}, we know that differentiating composite functions is relatively straightforward.
In particular, we have \cref{eq:chainrule} (the \enquote{Chain Rule}), which tells us that
\begin{equation*}
  \D{}{x} f \pbrac*{g \pbrac{x}} = f' \pbrac*{g \pbrac{x}} g' \pbrac{x}.
\end{equation*}
We can, of course, turn this into an antidifferentiation (and thus integration!) tool simply by antidifferentiating both sides.
Let's write that down for safekeeping.

\begin{theorem}
  \label{thm:subs.long}
  Suppose $f$ and $g$ are differentiable functions.
  Then the general antiderivative of $f' \pbrac*{g \pbrac{x}} g' \pbrac{x}$ is $f \pbrac*{g \pbrac{x}} + C$.
  In other words,
  \begin{equation}
    \label{eq:subs.long}
    \Int{f' \pbrac*{g \pbrac{x}} g' \pbrac{x}}{x} = f \pbrac*{g \pbrac{x}}.
  \end{equation}
\end{theorem}

We can use \cref{thm:subs.long} to compute the integral of any function that \enquote{looks like} the outcome of the chain rule.
Let's try an example to start to get a feel for how this works.

\begin{example}
  \label{ex:subs.simple.long}
  Find $\Int{3x^{2} \sqrt{x^{3} + 3}}{x}$.
\end{example}

\begin{soln}
  We note that $\D{}{x} x^{3} + 3 = 3 x^{2}$.
  This is good news!
  It suggests that we should set $g(x) = x^{3} + 3$; then $g'(x) = 3x^{2}$.
  If we let $f'(x) = \sqrt{x}$, we can rewrite our integrated: $3x^{2} \sqrt{x^{3} + 3} = g' \pbrac{x} f' \pbrac*{g \pbrac{x}}$.
  This is exactly the form required by \cref{thm:subs.long}!
  We can apply that result, once we find an antiderivative of $f'$: $f(x) = \frac{2}{3} x^{\sfrac{3}{2}}$ will do nicely.
  Then we can compute that
  \begin{equation*}
    \Int{3x^{2} \sqrt{x^{3} + 3}}{x} = \Int{g' \pbrac{x} f' \pbrac*{g \pbrac{x}}}{x} = f \pbrac*{g \pbrac{x}} + C = \frac{2}{3} \pbrac*{x^{3} + 3}^{\sfrac{3}{2}} + C.
  \end{equation*}
\end{soln}

\subsection{Change of variables and definite integrals}
\label{sec:int.subs.change-of-variables}
Ultimately, we'd like to use this tool to compute definite integrals.
These kinds of calculations can get pretty messy, with all those $f$s and $g$s and $'$s laying around to trip us up.
Let's take a moment to dig in and see if we can clean things up a bit.

Suppose we want to compute a definite integral using substitution---that is, we're looking for the value $\Int{f' \pbrac*{g \pbrac{x}} g' \pbrac{x}}{x, x=a, x=b}$.
(We write $x = a$ and $x = b$ as the limits of integration to emphasize that these are values of $x$; the significance of this will become evident shortly.)
By \cref{thm:subs.long,thm:ftc.calc}, this is given by $f \pbrac*{g \pbrac{b}} - f \pbrac*{g \pbrac{a}}$.
But this is the same value we would have obtained by integrating $f$ over the interval $\closedint{g \pbrac{a}, g \pbrac{b}}$!
Thus, we can instead compute this integral by \enquote{change of variables}.

\begin{theorem}
  \label{thm:subs.change-of-variables}
  Suppose $f$ and $g$ are differentiable functions.
  Then
  \begin{equation}
    \label{eq:subs.change-of-variables}
    \Int{f' \pbrac*{g \pbrac{x}}}{x, x=a, x=b} = f \pbrac*{g \pbrac{b}} - f \pbrac*{g \pbrac{a}} = \Int{f' \pbrac{u}}{u, u = g(a), u = g(b)}.
  \end{equation}
\end{theorem}

This notion of change of variables suggests a new form of \cref{thm:subs.long} which incorporates the intermediate variable $u$.
\begin{theorem}
  \label{thm:subs.short}
  Suppose $f$ and $g$ are differentiable functions.
  Let $u = g(x)$ and $\diff{u} = g'(x) \, \diff{x}$.
  Then
  \begin{equation}
    \label{eq:subs.short}
    \Int{f' \pbrac*{g \pbrac{x}}}{x} = \Int{f' \pbrac*{u}}{u} = f \pbrac{u} + C = f \pbrac{g \pbrac{x}} + C.
  \end{equation}
  This process is known as \enquote{$u$-substitution}.
\end{theorem}

\begin{note*}
  When applying \cref{thm:subs.short}, it is essential to keep track of the difference between the original variable $x$ and the new variable $u$!
\end{note*}

Let's give this a try with a couple of examples.
First, we'll revisit \cref{ex:subs.simple.short} using $u$-substitution.

\begin{example}
  \label{ex:subs.simple.short}
  Find $\Int{3x^{2} \sqrt{x^{3} + 3}}{x}$ using $u$-substitution.
\end{example}

\begin{soln}
  Set $u = x^{3} + 3$; then $\diff{u} = 3x^{2} \, \diff{x}$.
  Thus, applying \cref{eq:subs.short} and reordering our original integrand, we have that
  \begin{equation*}
    \int \underbrace{\sqrt{x^{3} + 3}}_{\sqrt{u}} \underbrace{3x^{2} \, \diff{x}}_{\diff{u}} = \Int{\sqrt{u}}{u} = \frac{2}{3} u^{\sfrac{3}{2}} + C = \frac{2}{3} \pbrac*{x^{3} + 3}^{\sfrac{3}{2}} + C.
  \end{equation*}
  Pleasingly, this agrees with the result of \cref{ex:subs.simple.long}!
\end{soln}

Now let's try a new integral.

\begin{example}
  \label{ex:subs.constantmult}
  Find $\Int{x e^{x^{2}}}{x}$ using $u$-substitution.
\end{example}

\begin{soln}
  This seems like a natural candidate for substitution with $u = e^{x^{2}}$.
  However, there's a slight problem---we can compute $\diff{u} = 2x \, \diff{x}$, but then we don't quite have this term in our original integral!
  The solution to this seeming conundrum is to look for what we \emph{do} have, which in this case is $x \, \diff{x} = \frac{1}{2} \diff{u}$.
  Thus, we can still rewrite our integral in terms of $\diff{u}$ and some function of $u$, which means we're in business!
  All that remains is to compute:
  \begin{equation*}
    \int \underbrace{e^{x^{2}}}_{e^{u}} \underbrace{x \, \diff{x}}_{\frac{1}{2} \diff{u}} = \Int{\frac{1}{2} e^{u}}{u} = \frac{1}{2} e^{u} + C = \frac{1}{2} e^{x^{2}} + C.
  \end{equation*}
\end{soln}

Of course, we can use this technique to compute definite integrals as well!

\begin{example}
  \label{ex:subs.definite}
  Find $\Int{\frac{2x}{x^{2} + 1}}{x, 0, 1}$.
\end{example}

\begin{soln}
  It's evident that we should use the substitution $u = x^{2} + 1$.
  Then $\diff{u} = 2x \, \diff{x}$, and we can proceed, using \cref{eq:subs.change-of-variables} to handle the limits of integration.
  \begin{multline*}
    \Int{\frac{2x}{x^{2} + 1}}{x, 0, 1} = \Int{\frac{1}{u}}{u, u(0), u(1)} = \ln \abs{u(1)} - \ln \abs{u(0)} \\
    = \ln \abs{1^{2} + 1} - \ln \abs{0^{2} + 1} = \ln 2 - \ln 1 = \ln 2.
  \end{multline*}
\end{soln}

\subsection{Substitution trickery}
\label{sec:int.subs.tricks}

So far, we've only considered applications of substitution where the choices of $f'$, $u$, and $\diff{u}$ are relatively obvious.
However, some of the most interesting and applications occur when this is not the case.
We'll close out this section by considering some examples of this phenomenon.

First, let's consider a case where there are several choices for $u$, some of which are better than others.

\begin{example}
  \label{ex:subs.hard-u}
  Find $\Int{\frac{x \cos x^{2}}{\sqrt{\sin x^{2}}}}{x}$ using $u$-substitution.
\end{example}

\begin{soln}
  One natural choice of $u$ in this problem is $u = x^{2}$.
  Then $\diff{u} = 2x \, \diff{x}$, which does in fact appear in our integral (after correcting for the constant $2$, anyway)!
  However, this leaves us with the rather unpleasant-looking integral $\Int{\frac{\cos u}{\sqrt{\sin u}}}{u}$.

  To make things more manageable, we need to make a different choice.
  This time, let $u = \sin x^{2}$.
  Then $\diff{u} = 2x \cos x^{2} \, \diff{x}$, which again appears in our original integral (after correcting for the constant $2$).
  This time, the resulting integral is much more approachable!
  All that remains is to compute:
  \begin{equation*}
    \Int{\frac{x \cos x^{2}}{\sqrt{\sin x^{2}}}}{x} = \Int{\frac{1}{\sqrt{u}} \frac{1}{2}}{u} = \sqrt{u} + C = \sqrt{\sin x^{2}} + C.
  \end{equation*}
\end{soln}

Next, we'll consider a couple of cases where we need to manipulate the integral in order to find the $f'(u)$ part after the substitution.

\begin{example}
  \label{ex:subs.easy-rewrite}
  Find $\Int{x \sqrt{x - 4}}{x}$.
\end{example}

\begin{soln}
  The substitution $u = x - 4$ seems reasonable---after all, it simplifies the expression under the radical, and $\diff{u} = \diff{x}$ is easy enough to find!
  However, this leaves us with the question of what to do with the leftover $x$.
  To handle this, we note that we can solve $u = x - 4$ for $x = u + 4$.
  Thus, we can rewrite our integral and then apply the substitution:
  \begin{multline*}
    \Int{x \sqrt{x - 4}}{x} = \Int{\pbrac*{\pbrac{x - 4} + 4} \sqrt{x - 4}}{x} = \\ \Int{\pbrac{u + 4} \sqrt{u}}{u} = \Int{u \sqrt{u} + 4 \sqrt{u}}{u} = \frac{2}{5} u^{\sfrac{5}{2}} + 4 \frac{2}{3} u^{\sfrac{3}{2}} + C \\
    = \frac{2}{5} \pbrac{x - 4}^{\sfrac{5}{2}} + \frac{8}{3} \pbrac{x - 4}^{\sfrac{3}{2}} + C.
  \end{multline*}
\end{soln}

This technique can also be applied in much messier settings.

\begin{example}
  \label{ex:subs.hard-rewrite}
  Find $\Int{\sqrt{1 + \sqrt{x}}}{x}$ using $u$-substitution.
\end{example}

\begin{soln}
  Let $u = 1 + \sqrt{x}$; then $\diff{u} = \frac{1}{2 \sqrt{x}} \diff{x}$.
  Our integrand is evidently $\sqrt{u}$, but where is the $\diff{u}$?
  It turns out that the way to proceed here is simply to \emph{make} $\diff{u}$ appear in the integrand, by solving for $\diff{x}$ and substituting the result.
  Since $\diff{u} = \frac{1}{2 \sqrt{x}} \diff{x}$, we can solve to find that $\diff{x} = 2 \sqrt{x} \, \diff{u}$.
  We defined $u = 1 + \sqrt{x}$, so $2 \sqrt{x} = 2u - 2$; thus, $\diff{x} = \pbrac{2u - 2} \, \diff{u}$.
  We can now substitute $u$ and $\diff{x}$ into our original integral:
  \begin{multline*}
    \Int{\sqrt{1 + \sqrt{x}}}{x} = \Int{\sqrt{u} \pbrac{2u - 2}}{u} = \Int{2u \sqrt{u} - 2 \sqrt{u}}{u} \\
    = 2 \cdot \frac{2}{5} u^{\sfrac{5}{2}} - 2 \cdot \frac{2}{3} u^{\sfrac{3}{2}} + C = \frac{4}{5} \pbrac*{1 + \sqrt{x}}^{\sfrac{5}{2}} - \frac{4}{3} \pbrac*{1 + \sqrt{x}}^{\sfrac{3}{2}} + C.
  \end{multline*}
\end{soln}

\begin{exercises}
\end{exercises}
\end{document}

%%% Local Variables:
%%% mode: latex
%%% TeX-master: "../book/calcnotes.tex"
%%% End: