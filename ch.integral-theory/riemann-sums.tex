\documentclass[../book/calcnotes.tex]{subfiles}

\begin{document}
\section{Riemann sums and numerical integration}
\label{sec:riemann-sums}

The theory we developed in \cref{sec:ftc}---and especially the result of \cref{thm:ftc.calc}---represent remarkable progress on our original problem.
Given a function $f$, we can now compute the value of $\Int{f(x)}{x, a, b}$ simply by evaluating some antiderivative of $f$ at two points!
When this is possible, it should be relatively straightforward to compute---although of course the process of \emph{finding} such an antiderivative may be complicated!
(Indeed, the art of finding antiderivatives occupies most of the rest of this chapter.)

However, there are many situations in which this is not possible.
\begin{example*}
  The \enquote{Gaussian function} $f(x) = e^{-x^{2}}$ has many applications in the theory of probability, where it is important to be able to compute integrals $\Int{e^{-t^{2}}}{t, 0, x}$.
  However, this function has no antiderivative in elementary functions!%
  \sidefootnote{The \emph{elementary functions} are those produced from power, exponential, logarithmic, trigonometric, and constant functions by composition, multiplication, division, addition, and subtraction. In other words, they're the \enquote{normal} functions that we've been studying all along. Naturally, the proof that $e^{-x^{2}}$ has \emph{no} elementary antiderivative is quite technical!}
  Thus, we need to take another path to compute these integrals.
\end{example*}

\begin{example*}
  In many \enquote{real-world} applications, we need to compute an integral $\Int{f(x)}{x, a, b}$ of a function $f$ for which we have no analytic expression.
  In particular, we may only have sample values of $f$, which could represent the amplitude of a sound wave (sampled at \SI{44.1}{\kilo\hertz} with a microphone and digital recorder) or the acceleration of a car (measured with an accelerometer every \SI{10}{\milli\second}).
  In these cases, it is not possible to find an exact antiderivative of $f$.
\end{example*}

To resolve this problem, we'll develop an alternative approach to integration which returns to the original \enquote{signed area} idea from \cref{def:defint}.
Consider a function $f$ whose values we know on $\closedint{a, b}$.
How can we estimate $\Int{f(x)}{x, a, b}$ without using any analytic information about $f$?

Let's try overlaying the region under $y = f(x)$ with some rectangles, as illustrated in \cref{fig:riemannleftex}.
This isn't a great approximation, but it has one important feature: the signed area will be \emph{very} easy to compute.
Let's try to work out what it will be (at least as well as we can without knowing what the function $f$ is).

\begin{medfig}
  \begin{tikzpicture}
    \begin{axis}[
      agdplot,
      rectify,
      ymin = -.3,
      ymax = .6,
      xmin = -.1,
      xmax = 5.5,
      ytick = \empty,
      xtick = \empty,
      extra x ticks = {1,2,3,4,5},
      extra x tick labels = {$a$,$x_{1}$, $x_{2}$, $x_{3}$, $b$},
      ]
      \pgfmathdeclarefunction{ftcfunc}{1}{\pgfmathparse{1/16+1/4*sin(deg(pi*#1^2/8))}}

      \addplot [leftriemann=1:5, integral segments=4, box1] {ftcfunc(x)};

      \addplot [thick, domain=-1:8] {ftcfunc(x)} node [pos=0.50,pin={150:$y = f(x)$}] {};
    \end{axis}
  \end{tikzpicture}
  \caption{Demonstration of approximation of an integral with four rectangles at left endpoints}
  \label{fig:riemannleftex}
\end{medfig}

To produce \cref{fig:riemannleftex}, we split the region $\closedint{a, b}$ into four equal subintervals, each of width $\frac{b - a}{4}$.
The height of each interval is determined by the value of $f$ at the left endpoint of the interval; we'll denote the $i$th of these left endpoints by $x_{i}^{*}$.
Then the signed area of the $i$th rectangle is just $f \pbrac*{x_{i}^{*}} \cdot \frac{b - a}{4}$, so the total signed area of the four-rectangle approximation is given by
\begin{equation}
  \label{eq:riemannleftex}
  A = \Sum{f \pbrac*{x_{i}^{*}} \cdot \frac{b - a}{4}}{i, 1, 4}.
\end{equation}
If we instead wanted to use $5$ or $72$ rectangles (or, more generally, some positive integer $n$), we'd merely need to replace both occurrences of $4$ in \cref{eq:riemannleftex} with $n$; that's the magic of sigma notation!

Of course, our choice to use the left endpoints of the intervals for the values $x_{i}^{*}$ was completely arbitrary.
If we want our approximation to make sense, the heights of the rectangles clearly should be given by values of $f$---but which ones?
There are plenty of reasonable choices: $x_{i}^{*}$ could be the left endpoint, the right endpoint, the midpoint, or even a randomly chosen point from the $i$th subinterval.
The form of \cref{eq:riemannleftex} won't change---only the procedure for choosing the values $x_{i}^{*}$.

Let's take a moment to pin down exactly what the values of $x_{i}^{*}$ will be in each of these cases.
Suppose we partition the interval $\closedint{a, b}$ into $n$ subintervals of width $\frac{b - a}{n}$, which we'll denote by $\Delta x$ for convenience..
Let the endpoints of these intervals be $x_{0}, x_{1}, \dots, x_{n}$, where $x_{0} = a$ and $x_{n} = b$.
The spacing between them is given by $\frac{b - a}{n} = \Delta x$, so clearly $x_{i} = a + i \Delta x$.
\begin{itemize}
\item
  The left endpoint of subinterval $i$ is $x_{i - 1} = a + \pbrac*{i-1} \Delta x$.

\item
  The right endpoint of subinterval $i$ is $x_{i} = a + i \Delta x$.

\item
  The midpoint of subinterval $i$ is $\frac{1}{2} \pbrac*{x_{i-1} + x_{i}} = a + \pbrac*{i - \sfrac{1}{2}} \Delta x$.
\end{itemize}

From each of these choices of $x_{i}^{*}$ we obtain an approximation of $\Int{f(x)}{x, a, b}$.
A left-endpoint approximation of an integral using four rectangles was illustrated in \cref{fig:riemannleftex}; the corresponding right-endpoint approximation is shown in \cref{fig:riemannrightex} and the midpoint approximation is shown in \cref{fig:riemannmidex}.

\begin{medfig}
  \begin{tikzpicture}
    \begin{axis}[
      agdplot,
      rectify,
      ymin = -.3,
      ymax = .6,
      xmin = -.1,
      xmax = 5.5,
      ytick = \empty,
      xtick = \empty,
      extra x ticks = {1,2,3,4,5},
      extra x tick labels = {$a$,$x_{1}$, $x_{2}$, $x_{3}$, $b$},
      ]
      \pgfmathdeclarefunction{ftcfunc}{1}{\pgfmathparse{1/16+1/4*sin(deg(pi*#1^2/8))}}

      \addplot [rightriemann=1:5, integral segments=4, box2] {ftcfunc(x)};

      \addplot [thick, domain=-1:8] {ftcfunc(x)} node [pos=0.50,pin={150:$y = f(x)$}] {};
    \end{axis}
  \end{tikzpicture}
  \caption{Demonstration of approximation of an integral with four rectangles at right endpoints}
  \label{fig:riemannrightex}
\end{medfig}

\begin{medfig}
  \begin{tikzpicture}
    \begin{axis}[
      agdplot,
      rectify,
      ymin = -.3,
      ymax = .6,
      xmin = -.1,
      xmax = 5.5,
      ytick = \empty,
      xtick = \empty,
      extra x ticks = {1,2,3,4,5},
      extra x tick labels = {$a$,$x_{1}$, $x_{2}$, $x_{3}$, $b$},
      ]
      \pgfmathdeclarefunction{ftcfunc}{1}{\pgfmathparse{1/16+1/4*sin(deg(pi*#1^2/8))}}

      \addplot [midriemann=1:5, integral segments=4, box3] {ftcfunc(x)};

      \addplot [thick, domain=-1:8] {ftcfunc(x)} node [pos=0.50,pin={150:$y = f(x)$}] {};
    \end{axis}
  \end{tikzpicture}
  \caption{Demonstration of approximation of an integral with four rectangles at midpoints}
  \label{fig:riemannmidex}
\end{medfig}

These three approximations will be very useful, so let's write them all down in one place for future reference.

\begin{definition}
  \label{def:riemannsum}
  Let $f$ be defined over $\closedint{a, b}$, fix an integer $n \geq 1$ (the number of subintervals), and let $\Delta x = \frac{b - a}{n}$ (the width of each subinterval).
  Then the \emph{$n$th left, right, and midpoint Riemann sums of $f$ over $\closedint{a, b}$} are $L_{n}$, $R_{n}$, and $M_{n}$, given by the following:
  \begin{subequations}
    \label{eq:riemanndef}
    \begin{align}
      L_{n} &= \Sum{f \pbrac*{a + \pbrac*{i-1} \Delta x} \Delta x}{i, 1, n} \label{eq:riemanndef.left} \\
      R_{n} &= \Sum{f \pbrac*{a + i \Delta x} \Delta x}{i, 1, n} \label{eq:riemanndef.right} \\
      M_{n} &= \Sum{f \pbrac*{a + \pbrac*{i - \sfrac{1}{2}} \Delta x} \Delta x}{i, 1, n} \label{eq:riemanndef.mid}
    \end{align}
  \end{subequations}
\end{definition}

Of course, an approximation wouldn't be any use if we couldn't be sure that it was good.
Fortunately, it turns out that Riemann sums actually do approximate integrals quite well.
In particular, as $n$ gets very large, the values of $L_{n}$, $R_{n}$, and $M_{n}$ all converge on the actual value of the associated integral.

\begin{theorem}
  \label{thm:riemannint}
  Let $L_{n}$, $R_{n}$, and $M_{n}$ be the Riemann sums of an integrable function $f$ over an interval $\closedint{a, b}$ in the sense of \cref{def:riemannsum}.
  Then\sidefootnote{Many authors, including Riemann himself, use this approach as their \emph{definition} of the definite integral.}
  \begin{equation}
    \label{eq:riemannint}
    \lim_{n \to \infty} L_{n} = \lim_{n \to \infty} R_{n} = \lim_{n \to \infty} M_{n} = \Int{f(x)}{x, a, b}.
  \end{equation}
\end{theorem}

So, any time we compute a finite Riemann sum, we're estimating the value of a definite integral!
Before we look at the many important applications of this idea, let's get a bit of practice computing these Riemann sums.

\begin{example}
  \label{ex:riemannx2}
  Let $f(x) = x^{2}$.
  Compute the fourth right-endpoint Riemann sum of $f$ over $\closedint{1, 3}$.
  Compare the result to the value of $\Int{f(x)}{x, 1, 3}$.
\end{example}

\begin{soln}
  We need to construct four rectangles over the interval $\closedint{1, 3}$ whose heights are the values of $f(x)$ at $x = 1.5$, $x = 2$, $x = 2.5$, and $x = 3$.
  This is illustrated in \cref{fig:riemannx2}.

  \begin{marginfigure}
    \centering
    \begin{tikzpicture}
      \begin{axis}[
        agdplot,
        xmin= -.1,
        xmax = 3.3,
        ymin = -.4,
        ymax = 10,
        ytick = \empty,
        xtick = {1,3},
        ]
        \pgfmathdeclarefunction{func}{1}{\pgfmathparse{#1^2}}

        \addplot [rightriemann=1:3, integral segments=4, box1] {func(x)};
        \addplot [thick, domain=0:4] {func(x)} node [pos=0.45,pin={130:$y = x^{2}$}] {};

      \end{axis}
    \end{tikzpicture}
    \caption{$R_{4}$ for $f(x) = x^{2}$ over $\closedint{1,3}$}
    \label{fig:riemannx2}
  \end{marginfigure}
  Since $f(x) = x^{2}$, we can easily compute the appropriate heights.
  These values, along with the corresponding values of $\Delta x$ and the resulting areas of the rectangles, are shown in \cref{tab:riemann2x}.
  From these, we can calculate that $R_{4} = \frac{9}{8} + 2 + \frac{25}{8} + \frac{9}{2} = \frac{43}{4} = 10.75$.

  \begin{table}[H]
    \centering
    \begin{tabular}{c c c c}
      \toprule
      $x$ & $f(x)$ & $\Delta x$ & Area \\
      \midrule
      $\sfrac{3}{2}$ & $\sfrac{9}{4}$ & $\sfrac{1}{2}$ & $\sfrac{9}{8}$ \\
      $2$ & $4$ & $\sfrac{1}{2}$ & $2$ \\
      $\sfrac{5}{2}$ & $\sfrac{25}{4}$ & $\sfrac{1}{2}$ & $\sfrac{25}{8}$ \\
      $3$ & $9$ & $\sfrac{1}{2}$ & $\sfrac{9}{2}$ \\ \bottomrule
    \end{tabular}
    \label{tab:riemann2x}
    \caption{Values for Riemann sum calculation in \cref{ex:riemannx2}}
  \end{table}

  We can also calculate $\Int{x^{2}}{x, 1, 3}$ exactly using \cref{thm:ftc.calc}.
  Since $F(x) + \frac{1}{3} x^{3}$ is an antiderivative of $f$, we have that $\Int{x^{2}}{x, 1, 3} = F(3) - F(1) = 9 - \frac{1}{3} = \frac{26}{3} = 8.\overline{66}$.
  These values are quite different---but, of course, we only used four rectangles!
\end{soln}

Now let's consider a situation where the Riemann sum is really necessary.

\begin{example}
  \label{ex:riemannex.car}
  A car accelerates on a racetrack.
  Its velocity is measured every second by an onboard computer, which records the values shown in \cref{tab:riemannex.car}.
  How far has the car gone after six seconds, in your best estimation?

  \begin{table}[H]
    \centering
    \begin{tabular}{c c}
      \toprule
      $t$ (\si{\second}) & $v$ (\si{\meter\per\second}) \\
      \midrule
      0 & 0 \\
      1 & 3 \\
      2 & 8 \\
      3 & 15 \\
      4 & 22 \\
      5 & 25 \\
      6 & 27 \\
      \bottomrule
    \end{tabular}
    \caption{Velocity of a car}
    \label{tab:riemannex.car}
  \end{table}
\end{example}

\begin{soln}
  We'd like to compute $\Int{v(t)}{t, 0, 6}$.
  However, we only know discrete values of $v$, so we have no hope of finding an antideriv4ative!
  Thus, we're going to need a Riemann sum.
  Over each \SI{1}{\second} time interval, the distance traveled by the car is approximated by $v \pbrac{t_{i}^{*}} \cdot \SI{1}{\second}$, where $t_{i}^{*}$ is a time during that interval.
  We can compute the values of $L_{6}$ and $R_{6}$ for this car using \cref{eq:riemanndef.left,eq:riemanndef.right} as follows:
  \begin{gather*}
    L_{6} = \SI{0}{\meter\per\second} \cdot \SI{1}{\second} + \SI{3}{\meter\per\second} \cdot \SI{1}{\second} + \SI{8}{\meter\per\second} \cdot \SI{1}{\second} + \SI{15}{\meter\per\second} \cdot \SI{1}{\second} + \SI{22}{\meter\per\second} \cdot \SI{1}{\second} + \SI{25}{\meter\per\second} \cdot \SI{1}{\second} = \SI{73}{\meter} \\
    R_{6} = \SI{3}{\meter\per\second} \cdot \SI{1}{\second} + \SI{8}{\meter\per\second} \cdot \SI{1}{\second} + \SI{15}{\meter\per\second} \cdot \SI{1}{\second} + \SI{22}{\meter\per\second} \cdot \SI{1}{\second} + \SI{25}{\meter\per\second} \cdot \SI{1}{\second} + \SI{27}{\meter\per\second} \cdot \SI{1}{\second} = \SI{100}{\meter} \\
  \end{gather*}
  So how far did the car go?
  Presumably, something between \SI{73}{\meter} and \SI{100}{\meter}.
  In fact, their average, \SI{86.5}{\meter}, is probably a better estimate than either $L_{6}$ or $R_{6}$, so let's call that our best estimation.
\end{soln}

\begin{exercises}
\end{exercises}
\end{document}

%%% Local Variables:
%%% mode: latex
%%% TeX-master: "../book/calcnotes.tex"
%%% End: