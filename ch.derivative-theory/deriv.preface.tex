\documentclass[../book/calcnotes.tex]{subfiles}

\begin{document}
\section*{Preface}
\label{sec:deriv.preface}
We'll begin by considering a simple-sounding problem.

\begin{motprob}
  Given a curve $y = f(x)$ and a point $\pbrac{x, y}$ on that curve, what is the tangent line to the curve at the point?
\end{motprob}

This problem has a very long history; it was considered in various forms by mathematicians as far back as Euclid and Archimedes in the third century BCE.
The analytic geometers of seventeenth-century Europe were particularly fascinated by this problem; Descartes, Barrow, and Fermat all invested significant time in it.
Ultimately, it was the development of differential calculus by Newton and Leibniz which allowed the solution of this problem.

We won't follow their work exactly---the intervening centuries have given us a lot of time to reflect on how best to approach this material.
Nevertheless, we'll follow the lead of these great thinkers as we begin our investigation.
\end{document}

%%% Local Variables:
%%% mode: latex
%%% TeX-master: "../book/calcnotes.tex"
%%% End: