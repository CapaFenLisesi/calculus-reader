\documentclass[../book/calcnotes.tex]{subfiles}

\begin{document}
\section{Antiderivatives}
\label{sec:antiderivs}

\begin{definition}
  \label{def:antideriv}
  Let $f$ and $F$ be functions which are continuous on an interval $I = \closedint{a, b}$.
  Then $F$ is an \emph{antiderivative of $f$ on $I$} if $\D{F}{x} = f \pbrac{x}$ on $I$.
\end{definition}

\begin{theorem}
  \label{thm:antideriv.constant}
  Let $F$ and $G$ be antiderivatives of a continuous function $F$.
  Then there is some constant $C \in \RR$ such that $F(x) = G(x) + C$.
  In other words, if $f$ is a continuous function, all its antiderivatives are equal up to a constant term.
\end{theorem}

\exercises
\end{document}

%%% Local Variables:
%%% mode: latex
%%% TeX-master: "../book/calcnotes.tex"
%%% End: