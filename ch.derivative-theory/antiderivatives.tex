\documentclass[../book/calcnotes.tex]{subfiles}

\begin{document}
\section{Antiderivatives}
\label{sec:antiderivatives}

\begin{definition}
  \label{def:antiderivative}
  Let $f$ and $F$ be functions which are continuous on an interval $I = \closedint{a, b}$.
  Then $F$ is an \emph{antiderivative of $f$ on $I$} if $\D{F}{x} = f \pbrac{x}$ on $I$.
\end{definition}

\begin{theorem}
  \label{thm:antiderivative.constant}
  Let $F$ and $G$ be antiderivatives of a continuous function $F$.
  Then there is some constant $C \in \RR$ such that $F(x) = G(x) + C$.
  In other words, if $f$ is a continuous function, all its antiderivatives are equal up to a constant term.
\end{theorem}

\begin{definition}
  \label{def:antiderivative.general}
  Let $F$ be an antiderivative of a continuous function $f$.
  Then the \emph{general antiderivative of $f$} is the one-parameter family of functions $\AD{}{x} f(x) = F(x) + C$ where $C$ takes values from $\RR$.
\end{definition}

\begin{note}
  In many settings, the general antiderivative of $f$ is denoted by $\Int{f(x)}{x}$ instead of $\AD{}{x} f(x)$, for reasons that will become clear in \cref{sec:integral-theory}.
  We'll stick with the $\AD{f}{x}$ notation for now, but you should be aware of this other one in case you see it show up in other settings.
\end{note}

\begin{exercises}
\end{exercises}
\end{document}

%%% Local Variables:
%%% mode: latex
%%% TeX-master: "../book/calcnotes.tex"
%%% End: