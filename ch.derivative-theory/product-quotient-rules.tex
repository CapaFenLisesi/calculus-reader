\documentclass[../book/calcnotes.tex]{subfiles}

\begin{document}
\section{The product  and quotient rules}
\label{sec:deriv.prodquot}

Back in \cref{sec:derivprops}, we saw how several algebraic operations on functions interact with differentiation.
In particular, we showed that, if we know the derivatives of two functions $f$ and $g$ and pick a constant $c$, we can use them to find the derivatives of the functions $c$, $cf$, and $f+g$.
Of course, there's more to algebra than sums and constant multiples!
What happens, for example, if we take the \emph{product} of two functions?


\subsection{Derivatives of products}
\label{sec:productrule}

To get ourselves in the right frame of mind, let's consider an example where we might be interested in the derivative of a product.
\begin{example}
  \label{ex:stockproduct}
  Susan works for Mathr, a startup which pays her partially in equity (that is, in shares of the company's stock as opposed to cash).
  At any given time, the value $V$ of her stock is given by $V = S \cdot P$, where $S$ is the number of shares she owns and $P$ is the price each share would fetch on the market.
  Since all three of these terms ($V$, $S$, and $P$) may be functions of time, we have $V(t) = S(t) \cdot P(t)$.

  Now suppose that, at the start of fiscal year 2014, Susan owns ten thousand shares worth \SI{5}[\dollar]{} each, that the company is awarding her \SI{1000}{shares\per\year}, and that the value of each share is increasing by \SI{1}[\dollar]{\per\year} due to some recent positive press.
  In other words, $S(2014) = \SI{1000}{shares}$, $P(2014) = \SI{5}[\dollar]{}$, $S'(2014) = \SI{1000}{shares\per\year}$, and $P'(2014) = \SI{1}[\dollar]{\per\year}$.
  How quickly is the value of Susan's portfolio growing?
  (That is, what is $V'(2014)$?)

  There are two obvious factors that contribute to $V'$:
  \begin{itemize}
  \item
    Each of the $S(2014) = \SI{10000}{shares}$ that Susan owns is increasing in value by $P'(2014) = \SI{1}[\dollar]{\per\year}$.
    This contributes $S(2014) \cdot V'(2014) = \SI{10000}[\dollar]{\per\year}$ to $V'(2014)$.

  \item
    In addition, each of the $S'(2014) = \SI{1000}{shares\per\year}$ Mathr awards to Susan is worth $V(2014) = \SI{5}[\dollar]{}$.
    This contributes $S'(2014) \cdot V(2014) = \SI{5000}[\dollar]{\per\year}$ to $V'(2014)$.
  \end{itemize}

  These two terms together thus contribute a total of $S(2014) \cdot V'(2014) + S'(2014) \cdot V(2014) = \SI{15000}[\dollar]{\per\year}$ to the rate of growth $V'(2014)$ of Susan's portfolio value.
  Of course, the new shares are also growing in value even as Susan acquires them, so it seems like this calculation must be incomplete somehow---but it turns out\footnote{Essentially, this is because the contribution from this is \enquote{tiny} in a specific technical sense; working this out formally requires a pretty messy computation involving limits which we'll illustrate later.} that it isn't!
  Susan's portfolio really is growing at \SI{15000}[\dollar]{\per\year} at the start of fiscal year 2014.
\end{example}

Of course, all this talk about Susan and her money is just window dressing.
We really just argued that, for differentiable functions $S$ and $P$, their product function $V(t) = S(t) \cdot P(t)$ is differentiable, and furthermore that its derivative is given by $V'(t) = S(t) \cdot P'(t) + S'(t) \cdot P(t)$.
This turns out to hold in complete generality!
We should record it for future reference.

\begin{theorem}[Product rule for derivatives]
  \label{thm:deriv.product-rule}
  Let $f$ and $g$ be functions which are differentiable at $x$.
  Then $f \cdot g$ is differentiable at $x$, and
  \begin{equation}
    \label{eq:productrule}
    \D{}{x} \pbrac{f \cdot g} \pbrac{x} = f' \pbrac{x} \cdot g \pbrac{x} + f \pbrac{x} \cdot g' \pbrac{x}.
  \end{equation}
\end{theorem}

\begin{note}
  \label{note:fakeprodrule}
  In practice, it's tempting to ignore \cref{eq:productrule} and try to use the natural-seeming equation $\D{}{x} \pbrac{f \cdot g} \pbrac{x} \enquote{=} f' \pbrac{x} \cdot g' \pbrac{x}$ in its place.
  \emph{This is completely false!}
  Don't fall into the trap!
\end{note}

With \cref{thm:deriv.product-rule} in hand, we can compute the derivative of any function which decomposes as a product of functions whose derivatives we know.
Let's take a look at a couple of examples.

\begin{example}
  \label{ex:derivative.product.exp-poly}
  Find the derivative of $f(x) = e^{x} x^{3}$.
\end{example}

\begin{example}
  \label{ex:derivative.product.sin-log}
  Find the derivative of $f(x) = \sin x \cdot \ln x$.
\end{example}

\subsubsection{An analytic interpretation}
We can also prove \cref{thm:deriv.product-rule} analytically, using the original definition of the derivative (\cref{def:deriv.func}).

\begin{proof}[Proof of \cref{thm:deriv.product-rule}]
  Let $f$ and $g$ be functions which are differentiable at $x$.
  By \cref{def:deriv.func}, the derivative of their product $f \cdot g$ is given by
  \begin{align*}
    \pbrac*{f \cdot g}' \pbrac{x} =& \lim_{\Delta x \to 0} \frac{\pbrac*{f \cdot g} \pbrac*{x + \Delta x} - \pbrac*{f \cdot g} \pbrac{x}}{\Delta x} \\
    =& \lim_{\Delta x \to 0} \frac{f \pbrac*{x + \Delta x} \cdot g \pbrac*{x + \Delta x} - f \pbrac{x} \cdot g \pbrac{x}}{\Delta x}.
    \intertext{To make this more manageable, we'll use a bit of clever algebra, adding and subtracting $f \pbrac*{x + \Delta x} \cdot g \pbrac{x}$ to the numerator:}
    =& \lim_{\Delta x \to 0} \frac{f \pbrac*{x + \Delta x} \cdot g \pbrac*{x + \Delta x} - f \pbrac*{x + \Delta x} \cdot g \pbrac{x} + f \pbrac*{x + \Delta x} \cdot g \pbrac{x} - f \pbrac{x} \cdot g \pbrac{x}}{\Delta x}.
    \intertext{We can then break the right-hand side into two separate limits:}
    =& \lim_{\Delta x \to 0} \frac{f \pbrac*{x + \Delta x} \cdot g \pbrac*{x + \Delta x} - f \pbrac*{x + \Delta x} \cdot g \pbrac{x}}{\Delta x} \\
    &+ \lim_{\Delta x \to 0} \frac{f \pbrac*{x + \Delta x} \cdot g \pbrac{x} - f \pbrac{x} \cdot g \pbrac{x}}{\Delta x} \\
    =& \lim_{\Delta x \to 0} \frac{f \pbrac*{x + \Delta x} \cdot \pbrac*{g \pbrac*{x + \Delta x} - g \pbrac{x}}}{\Delta x} \\
    &+ \lim_{\Delta x \to 0} \frac{\pbrac*{f \pbrac*{x + \Delta x} - f \pbrac{x}} \cdot g \pbrac{x}}{\Delta x}. \\
    \intertext{Decomposing each term, we can write}
    =& \pbrac*{\lim_{\Delta x \to 0} f \pbrac*{x + \Delta x}} \cdot \pbrac*{\lim_{\Delta x \to 0} \frac{g \pbrac*{x + \Delta x} - g \pbrac{x}}{\Delta x}} \\
    &+ \pbrac*{\lim_{\Delta x \to 0} g \pbrac{x}} \cdot \pbrac*{\lim_{\Delta x \to 0} \frac{f \pbrac*{x + \Delta x} - f \pbrac{x}}{\Delta x}}.
    \intertext{The limits involving fractions should look very familiar: by \cref{def:deriv.func}, they are the derivatives of $g$ and $f$ at $x$ respectively!
      Moreover, $\lim_{\Delta x \to 0} f \pbrac*{x + \Delta x} = f \pbrac{x}$, because $f$ must be a continuous function to be differentiable.\footnotemark
      Thus,}
    =& f(x) \cdot g'(x) + f'(x) \cdot g(x).
  \end{align*}
  And this is exactly the result we want for \cref{thm:deriv.product-rule}!
  \footnotetext{We haven't defined continuity formally in this text, but this property is the definition mathematicians usually use; it's not hard to show that any differentiable function must be continuous.}
\end{proof}

% TODO: Write up geometric interpretation?

\subsection{Derivatives of quotients}
\label{sec:quotientrule}
\Cref{thm:deriv.product-rule} allows us to compute the derivative of the product $f \cdot g$ of two differentiable functions $f$ and $g$.
What about the quotient $\sfrac{f}{g}$?
Is it differentiable?
If so, what is its derivative?

It turns out, that in fact, this function \emph{is} differentiable, as long as $g(x) \neq 0$.
(Division by zero will continue to be a thorn in our side throughout the course.)
The result is straightforward enough to state:

\begin{theorem}[Quotient rule for derivatives]
  \label{thm:deriv.quotient-rule}
  Let $f$ and $g$ be functions which are differentiable at $x$ such that $g(x) \neq 0$.
  Then $\sfrac{f}{g}$ is differentiable at $x$, and
  \begin{equation}
    \label{eq:quotientrule}
    \D{}{x} \pbrac*{\frac{f}{g}} \pbrac{x} = \frac{f'(x) \cdot g(x) - f(x) \cdot g'(x)}{\pbrac*{g(x)}^{2}}.
  \end{equation}
\end{theorem}

\Cref{thm:deriv.quotient-rule} can be proved directly using algebraic trickery, just like we did above for \cref{thm:deriv.product-rule}.
However, it will be much easier to prove using the results we'll obtain in \cref{sec:chainrule}, so we'll postpone it until then.

\subsubsection{Derivatives of the remaining trig functions}
\label{sec:deriv.trig.other}
Back in \cref{sec:deriv.trig}, we showed (in \cref{thm:deriv.sincos}) that $\D{}{x} \sin x = \cos x$ and $\D{}{x} \cos x = -\sin x$.
It turns out that, with \cref{thm:deriv.quotient-rule} in hand, we can compute the derivatives of all the other trig functions using these two building blocks!
Let's take a look at how this works.

\begin{example}
  \label{ex:deriv.tan}
  Compute $\D{}{x} \tan x$.
\end{example}

\begin{soln}
  Recall that $\tan x = \frac{\sin x}{\cos x}$.
  Thus, by \cref{thm:deriv.quotient-rule},
  \begin{align*}
    \D{}{x} \tan x
    &= \D{}{x} \frac{\sin x}{\cos x} \\
    &= \frac{\pbrac*{\sin x}' \pbrac*{\cos x} - \pbrac*{\sin x} \pbrac*{\cos x}'}{\pbrac*{\cos x}^{2}}.
    \intertext{
      Using the derivatives we computed for $\sin x$ and $\cos x$, we continue:
    }
    &= \frac{\pbrac{\cos x}^{2} + \pbrac{\sin x}^{2}}{\pbrac{\cos x}^{2}}.
    \intertext{
      One of the most important trigonometric identities is that, for any angle $x$, $\sin^{2} x + \cos^{2} x = 1$!
      Thus,
    }
    &= \frac{1}{\cos^{2} x} = \sec^{2} x.
  \end{align*}
  So $\D{}{x} \tan x = \sec^{2} x$!
\end{soln}

We can use this same technique to compute the derivatives of the remaining three trigonometric functions.
We'll leave the rest of the computations as exercises, but we'll wrap up the results here:
\begin{theorem}
  \label{thm:deriv.othertrig}
  The derivatives of the tangent, cotangent, secant, and cosecant functions are given by
  \begin{align}
    \D{}{x} \tan x &= \sec^{2} x \label{eq:deriv.tan} \\
    \D{}{x} \cot x &= -\csc^{2} x \label{eq:deriv.cot} \\
    \D{}{x} \sec x &= \sec x \tan x \label{eq:deriv.sec} \\
    \D{}{x} \csc x &= -\csc x \cot x \label{eq:deriv.csc}
  \end{align}
\end{theorem}

\begin{exercises}
  \begin{exc}
    Prove \cref{eq:deriv.cot,eq:deriv.sec,eq:deriv.csc} using similar methods to those from \cref{ex:deriv.tan}.
  \end{exc}
\end{exercises}
\end{document}

%%% Local Variables:
%%% mode: latex
%%% TeX-master: "../book/calcnotes.tex"
%%% End: