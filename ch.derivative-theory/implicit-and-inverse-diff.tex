\documentclass[../book/calcnotes.tex]{subfiles}

\begin{document}
\section{Implicit differentiation and inverse functions}
\label{sec:derivs.implicit}

Way back in \cref{sec:deriv.definition}, we began our investigation of the derivative with a simple question: how can we find the tangent line to a curve at a point?
Of course, we focused our attention on curves of the form $y = f(x)$, and it turned out that there was quite a lot we could say about this problem.
However, many perfectly good plane curves are not given in this form.
For example, consider the curve $y^{2} = x^{3} - x + 1$, shown in \cref{fig:elliptic} along with its tangent line at $\pbrac{0, 1}$.
What is the slope of this line?

\begin{smallfig}
  \begin{asy}
    smallsize();

    real xstart=-2, xend=2.5;
    real ystart=-2, yend=2;

    import contour;
    real f2v(real x, real y) { return y^2 - (x^3 - x + 1); }
    guide[][] thegraphs = contour(f2v, a=(xstart,ystart), b=(xend,yend), new real[] {0});
    draw(thegraphs[0], thickplot);

    xaxis("$x$", xstart, xend, Arrow);
    yaxis("$y$", ystart, yend, Arrow);

    real f(real x) { return sqrt(x^3 - x + 1); }
    real df(real x) { return (3x^2-1)/(2*sqrt(x^3-x+1)); }
    tanplot(f, df, 0, xstart, xend, dashed);
  \end{asy}
  \caption{The curve $y^{2} = x^{3} - x + 1$ and its tangent line at $\pbrac{0, 1}$.}
  \label{fig:elliptic}
\end{smallfig}

If we knew an \enquote{explicit} function $f$ such that this was the curve $y = f(x)$, we'd be in business---we could apply the theory we've been developing to calculate its derivative, and that would tell us the slope.
However, we don't---and, what's more, we can't possibly, since this curve isn't the plot of a function\footnote{Note that, for most values of $x$, there are \emph{two} values of $y$ satisfying $y^{2} = x^{3} - x + 1y$.} at all!

Is there anything we can do?
Let's put aside our concern about $y$ not being a function of $x$ and just pretend\footnote{Formally, we can get away with this because $y$ is \enquote{locally} a function of $x$---that is, we can split the curve up into pieces which are plots of functions, which is good enough for what we're going to do.} that it is.
Whatever this function is, we know that it satisfies
\begin{equation*}
  y^{2} = x^{3} - x - 1
\end{equation*}
We'd like to learn something about the derivative $\D{y}{x}$ of this function; as a first step, let's try taking the derivative with respect to $x$ of each side of the defining equation to get
\begin{equation*}
  \D{}{x} y^{2} = \D{}{x} x^{3} - x - 1.
\end{equation*}
If $y$ is some function of $x$, by the Chain Rule (\cref{thm:chainrule}), we must have that $\D{}{x} \pbrac*{y(x)}^{2} = 2 y'(x) y(x)$.
Thus, we can actually take the derivatives on both sides of the equation above, so
\begin{equation*}
  2 y' y = 3x^{2} - 1.
\end{equation*}
Finally, we're really interested in finding values of $y'$, so we solve for this:
\begin{equation*}
  y' = \frac{3x^{2} - 1}{2y}.
\end{equation*}

So what have we achieved?
We have actually shown that, at a point $\pbrac{x, y}$, the tangent line to $y^{2} = x^{3} - x - 1$ has slope $y' = \frac{3x^{2} - 1}{2y}$.
This is actually exactly what we needed to solve our original problem!
All we need to do now is substitute $x = 0$ and $y = 1$ to find that the slope is $\frac{3 \pbrac{0}^{2} - 1}{2 \pbrac{1}} = -\frac{1}{2}$.

This technique is known as \emph{implicit differentiation}, and it turns out to have a wide variety of applications to other parts of calculus.
In particular, in \cref{sec:related-rates}, we'll see that it can be used in a large class of applied problems.
In addition, certain clever applications of implicit differentiation will allow us to find the derivatives of functions which were previously inaccessible.

\subsection{Derivatives of inverse functions}
\label{sec:derivs.inverse}
Back in \cref{sec:derivs.exp}, we discovered how to calculate the derivatives of exponential functions.
What about the equally-important logarithmic functions?

To start, let's recall that the logarithmic functions are defined to be the \emph{inverses} of the exponential functions.
In other words, for a given base $a > 0$, we have that $\log_{a} a^{x} = x = a^{\log_{a} x}$ for all $x$; in particular, $\ln e^{x} = x = e^{\ln x}$.
Can we use this fact to find $\D{}{x} \ln x$?

Let's have a go at it using the machinery of this section.
Let $y = \ln x$; then $x = e^{y}$.
Taking the derivative of both sides of this second equation, we find that $\D{}{x} x = \D{}{x} e^{y}$.
Applying \cref{thm:chainrule} to the right-hand side, we find that $1 = y' e^{y}$, so $y' = \frac{1}{e^{y}}$.
But $x = e^{y}$, so we can substitute this into the denominator to find that
\begin{equation*}
  \D{}{x} \ln x = y' = \frac{1}{e^{\ln x}} = \frac{1}{x}.
\end{equation*}

Weird!
We should definitely write that down.
(We'll go ahead and throw in the result for general logarithms, which is obtained in the same way.)

\begin{theorem}
  \label{thm:deriv.log}
  Fix a real base $a > 0$.
  Then the derivative of the logarithm function base $a$ is given by
  \begin{equation}
    \label{eq:deriv.log}
    \D{}{x} \log_{a} x = \frac{1}{x \ln a}.
  \end{equation}
  In particular,
  \begin{equation}
    \label{eq:deriv.log.natural}
    \D{}{x} \ln x = \frac{1}{x}.
  \end{equation}
\end{theorem}

Although we focused specifically on exponential and logarithmic functions, the line of argument we just used will actually work equally well for \emph{any} function and its inverse.
Let $f$ be a differentiable function with $g = f^{-1}$ (that is, $f(g(x)) = x = g(f(x))$ for all $x$).
If we let $y = g(x)$, we have that $x = f(y)$; differentiating both sides gives us that $\D{}{x} x = \D{}{x} f(y)$ and thus $1 = y' f'(y)$.
Solving for $y'$, we obtain that $y' = \frac{1}{f'(y)} = \frac{1}{f'(f(x))}$, just as before.

\begin{theorem}
  \label{thm:deriv.inverse}
  Let $f$ be a function which is differentiable and invertible.
  Then its inverse $f^{-1}$ is differentiable also, and
  \begin{equation}
    \label{eq:deriv.inverse}
    \D{}{x} f^{-1} (x) = \frac{1}{f' \pbrac*{f^{-1} \pbrac{x}}}.
  \end{equation}
\end{theorem}

With \cref{thm:deriv.inverse} in hand, we can compute the derivatives of many important inverse functions.
Let's consider some examples.

\begin{example}
  \label{ex:deriv.invtrig}
  Compute $\D{}{x} \sin^{-1} x$ and $\D{}{x} \cos^{-1} x$.
\end{example}

\begin{soln}
  We can simply apply \cref{thm:deriv.inverse} to the function $f(x) = \sin x$ (so $f'(x) = \cos x$) to compute that
  \begin{equation}
    \label{eq:deriv.arcsin.prelim}
    \D{}{x} \sin^{-1} \pbrac*{x} = \frac{1}{\cos \pbrac*{\sin^{-1} x}}.
  \end{equation}

  This is a nice result, but we actually can do much better.
  Let's let $\theta = \sin^{-1} x$ (that is, let $\theta$ be the angle in $\pbrac*{-\sfrac{\pi}{2}, \sfrac{\pi}{2}}$ such that $\sin \theta = x$).
  Then the denominator of the fraction in \cref{eq:deriv.arcsin.prelim} is equal to $\cos \theta$.
  If we're going to talk about sines and cosines of angles, we really ought to be looking at a right triangle, so consider the one shown in \cref{fig:deriv.arcsin.trig}.

  \begin{smallfig}
    \begin{asy}
      smallsize();

      import geometry;
      import math;

      real theta = 30;

      pair A = (0, 0);
      pair B = (1, 0);
      pair C = (1, Sin(theta));

      draw(A--B--C--cycle);

      label("$1$", A--C, LeftSide);
      label("$x$", B--C, RightSide);
      label("$\sqrt{1 - x^{2}}$", A--B, RightSide);

      markrightangle(A, B, C);
      markangle("$\theta$", B, A, C);
    \end{asy}
    \caption{A triangle such that $x = \sin \theta$}
    \label{fig:deriv.arcsin.trig}
  \end{smallfig}

  This triangle was carefully chosen so that its hypotenuse is $1$ and $x = \sin \theta$.
  We can then use the Pythagorean theorem to compute that $\cos \theta = \sqrt{1 - x^{2}} = \cos \pbrac*{\sin^{-1} x}$.
  Thus, we can rewrite \cref{eq:deriv.arcsin.prelim} without any reference to trigonometric functions at all:
  \begin{equation}
    \label{eq:deriv.arcsin}
    \D{}{x} \sin^{-1} x = \frac{1}{\sqrt{1 - x^{2}}}.
  \end{equation}

  Similarly, letting $f(x) = \cos x$ (so $f'(x) = -\sin x$), we compute that
  \begin{equation}
    \label{eq:deriv.arccos.prelim}
    \D{}{x} \cos^{-1} x = \frac{1}{-\sin \pbrac*{\cos^{-1} x}}.
  \end{equation}
  A similar calculation with a triangle like\footnote{Indeed, it is exactly the same triangle, with the name $\theta$ assigned to the other angle so that $x = \cos \theta$.} that in \cref{fig:deriv.arcsin.trig} reveals that
  \begin{equation}
    \label{eq:deriv.arccos}
    \D{}{x} \cos^{-1} x = \frac{-1}{\sqrt{1 - x^{2}}}.
  \end{equation}
\end{soln}

\subsection{Logarithmic differentiation}
\label{sec:deriv.logdiff}
Implicit differentiation also lets us simplify complicated derivative calculations involving products and exponents.
The general strategy will be to replace an equation of the form $y = f(x)$ with one like $\ln y = \ln f(x)$, manipulate the right-hand side using log laws, then differentiate both sides of the result.
This will be easiest to explain with an example.

\begin{example}
  \label{ex:deriv.logdiff}
  Let $f(x) = x^{3} \frac{\sqrt{x - 2}}{\sqrt{x + 1}}$.
  Calculate $f'(x)$.
\end{example}

\begin{soln}
  Of course, we \emph{could} use the Product and Quotient rules (\cref{thm:deriv.product-rule,thm:deriv.quotient-rule}) to calculate this derivative, but it's going to be a mess.
  Let's try the trick described above.
  Taking the logarithm of both sides of the definition of $f$, we find that
  \begin{align*}
    \ln f(x) &= \ln \pbrac*{x^{3} \frac{\sqrt{x - 2}}{\sqrt{x + 1}}}. \\
    \intertext{Applying some log laws, we can simplify this:}
    &= \ln \pbrac*{x^{3}} + \ln \pbrac*{\sqrt{x - 2}} - \ln \pbrac*{\sqrt{x + 1}} \\
    &= 3 \ln x + \frac{1}{2} \ln \pbrac*{x - 2} - \frac{1}{2} \ln \pbrac*{x + 1}.
  \end{align*}
  Differentiating both sides, we find that
  \begin{equation*}
    \D{}{x} \ln f(x) = \D{}{x} 3 \ln x + \frac{1}{2} \ln \pbrac*{x - 2} - \frac{1}{2} \ln \pbrac*{x + 1},
  \end{equation*}
  where the derivative of the right-hand side is now \emph{much} simpler to calculate than it was before.
  Thus, we can compute
  \begin{equation*}
    \frac{f'(x)}{f(x)} = \frac{3}{x} + \frac{1}{2 \pbrac*{x - 2}} - \frac{1}{2 \pbrac*{x + 1}}.
  \end{equation*}
  We already have a formula for $f(x)$, so we can solve for $f'(x)$ to conclude that
  \begin{equation*}
    f'(x) = \frac{\frac{3}{x} + \frac{1}{2 \pbrac*{x - 2}} - \frac{1}{2 \pbrac*{x + 1}}}{f(x)}
  \end{equation*}
  and so
  \begin{equation*}
    \D{}{x} x^{3} \frac{\sqrt{x - 2}}{\sqrt{x + 1}} = \frac{\frac{3}{x} + \frac{1}{2 \pbrac*{x - 2}} - \frac{1}{2 \pbrac*{x + 1}}}{x^{3} \frac{\sqrt{x - 2}}{\sqrt{x + 1}}}.
  \end{equation*}
\end{soln}

Whew!
That was a bit of a mess, but it's still an improvement over using the Product and Quotient rules directly.
The improvement becomes even more evident if we make the function more complicated, but we'll leave that for the exercises.


\begin{gps}
  \begin{gp}
    Use methods like those in \cref{ex:deriv.invtrig} to show that
    \begin{equation}
      \label{eq:deriv.arctan}
      \D{}{x} \tan^{-1} x = \frac{1}{1 + x^{2}}.
    \end{equation}
  \end{gp}

  \begin{gp}
    Calculate $\D{}{x} x^{x}$.
  \end{gp}
\end{gps}

\begin{exercises}
\end{exercises}
\end{document}

%%% Local Variables:
%%% mode: latex
%%% TeX-master: "../book/calcnotes.tex"
%%% End: