\documentclass[../book/calcnotes.tex]{subfiles}

\begin{document}
\section{Trigonometric derivatives}
\label{sec:deriv.trig}

Another important class of functions comes from \emph{trigonometry}, the study of the geometry of triangles.
\begin{smallfig}
  \begin{asy}
    smallsize();

    import geometry;
    import math;

    real theta = 40;

    pair a = (0, 0);
    pair b = (Cos(theta), 0);
    pair c = (Cos(theta), Sin(theta));

    draw(a--b--c--cycle);

    label("$C$", a--c, LeftSide);
    label("$A$", b--c, RightSide);
    label("$B$", a--b, RightSide);

    markrightangle(a, b, c);
    markangle("$\theta$", b, a, c);
  \end{asy}
  \caption{A triangle for trigonometry}
  \label{fig:trig.triangle}
\end{smallfig}
Given a right triangle like the one in \cref{fig:trig.triangle}, we define six trigonometric functions, corresponding to the six possible ratios of side lengths:
\begin{align*}
  \sin \theta &= \frac{A}{C} & \csc \theta &= \frac{C}{A} = \frac{1}{\sin \theta} \\
  \cos \theta &= \frac{B}{C} & \sec \theta &= \frac{C}{B} = \frac{1}{\cos \theta} \\
  \tan \theta &= \frac{A}{B} = \frac{\sin \theta}{\cos \theta} & \cot \theta &= \frac{B}{A} = \frac{1}{\tan \theta}
\end{align*}

Crucially, it turns out that these ratios depend only on the angle $\theta$ and not on the size of the triangle, so we can think of these as \emph{functions of $\theta$}.
The sine and cosine functions turn out to have particularly nice-looking graphs, as shown in \cref{fig:sin,fig:cos}.

\begin{medfig}
  \begin{asy}
    medsize();

    real f(real x) {return sin(x);}

    real xstart=-1, xend=13;
    real ystart=-1.2, yend=1.4;

    xaxis("$x$", xstart, xend,Arrow);
    yaxis("$y$", ystart, yend, Arrow);

    funcplot(f, xstart, xend);
  \end{asy}
  \caption{Plot of $y = \sin x$}
  \label{fig:sin}
\end{medfig}

\begin{medfig}
  \begin{asy}
    medsize();

    real f(real x) {return cos(x);}

    real xstart=-1, xend=13;
    real ystart=-1.2, yend=1.4;

    xaxis("$x$", xstart, xend,Arrow);
    yaxis("$y$", ystart, yend, Arrow);

    funcplot(f, xstart, xend);
  \end{asy}
  \caption{Plot of $y = \cos x$}
  \label{fig:cos}
\end{medfig}

These functions will turn out to be surprisingly important in calculus.
To see why, we'll need to reframe things a bit.

Before we dive into the calculus, let's revisit the definitions of the trigonometric functions.
Instead of thinking of the triangle from \cref{fig:trig.triangle}, let's put it inside a unit circle, as shown in \cref{fig:trig.circle.pre}.

\begin{medfig}
  \begin{asy}
    medsizescrunch();

    import geometry;
    import math;

    xaxis("$x$", -1.2, 1.2, Arrow);
    yaxis("$y$", -1.2, 1.2, Arrow);

    pair O = (0, 0);
    dot(0);
    label("$O$", O, SW);

    real theta = 40;

    pair B = (Cos(theta), 0);
    pair P = (Cos(theta), Sin(theta));

    draw(O--B--P--cycle);

    label("$1$", O--P, LeftSide);
    label("$x$", B--P, RightSide);
    label("$y$", O--B, RightSide);

    dot(B);
    label("$B$", B, S);

    dot(P);
    label("$P = \pbrac*{x, y}$", P, NE);

    markrightangle(O, B, P);
    markangle("$\theta$", B, O, P);

    draw(Circle((0,0),1));
  \end{asy}
  \caption{A circle for trigonometry}
  \label{fig:trig.circle.pre}
\end{medfig}

How do the values of $x = \cos \theta$ and $y = \sin \theta$ change as $\theta$ changes?
(In other words, what are $\D{x}{\theta}$ and $\D{y}{\theta}$?)

Suppose the point $P$ is moving counterclockwise around the circle with constant velocity $1$.
To visualize this, we might imagine that the point is being swung around $O$ on the end of a string of length $1$.
Let's take a closer look at this situation, shown in \cref{fig:trig.circle}.

Suppose that, at the instant shown in the picture, we cut the string, letting the point at $P$ fly away in a straight line.
$P$ will move in the direction \emph{perpendicular} to the line to $O$; in other words, its flight path is the line \emph{tangent} to the circle.
Since $P$ moves with velocity $1$, its position one time unit later is given by the upper triangle in \cref{fig:trig.circle}; the length of the vertical side is $\D{y}{t}$ and that of the horizontal side is $\D{x}{t}$, since we let it fly for exactly one time unit.

Now let's consider that triangle geometrically.
It's clearly a right triangle, since $x$ and $y$ are perpendicular coordinates.
Since $\angle O P Q$ is right, we know that $\angle Q P C$ is $\theta$.
Futhermore, by construction, the hypotenuse $\overline{P Q}$ has length $1$.
Thus, the triangle $\triangle O P B$ is congruent to the triangle $\triangle P Q C$.
This means that, in particular, $\D{x}{t} = y$ and $\D{y}{t} = x$.

\begin{medfig}
  \begin{asy}
    medsizescrunch();

    import geometry;
    import math;

    xaxis("$x$", -1.2, 1.2, Arrow);
    yaxis("$y$", -1.2, 1.2, Arrow);

    pair O = (0, 0);
    dot(0);
    label("$O$", O, SW);

    real theta = 40;

    label("$A$", arc(O, 1, 0, theta));

    pair B = (Cos(theta), 0);
    pair P = (Cos(theta), Sin(theta));

    pair C = P+(0, Cos(theta));
    pair Q = P+(-Sin(theta), Cos(theta));

    draw(O--B--P--cycle);

    label("$1$", O--P, LeftSide);
    label("$x(t)$", B--P, RightSide);
    label("$y(t)$", O--B, RightSide);

    draw(P--Q);
    draw(P--C);
    draw(C--Q);

    label("$1 = \D{A}{t}$", P--Q, RightSide);
    label("$\D{x}{t}$", C--Q, 2*RightSide);
    label("$\D{y}{t}$", P--C, RightSide);

    dot(P);
    label("$P = \pbrac*{x(t), y(t)}$", P, NE);

    dot(B);
    label("$B$", B, S);

    dot(Q);
    label("$Q$", Q, NW);

    dot(C);
    label("$C$", C, NE);

    markrightangle(O, B, P);
    markangle("$\theta$", B, O, P);

    markrightangle(P, C, Q);
    markrightangle(O, P, Q);
    markangle("$\theta$", C, P, Q);

    draw(Circle((0,0),1));

    limits((-.25,-.25), (1.2,2), Crop);
  \end{asy}
  \caption{Circular motion}
  \label{fig:trig.circle}
\end{medfig}

What does this have to do with $\theta$?
Remember that we initially set $P$ moving along the circle at velocity $1$.
If $A$ is the length of the arc along the circle from $(1, 0)$ to $P$, then this means that $\D{A}{t} = 1$; in particular, if $P$ started from $(1, 0)$ at $t = 0$, then $A = t$.
But in radian\footnote{In fact, this is the \emph{definition} of radians.} measure, $\theta = A$!
So $\theta = t$ also.
Thus, we can just substitute $\theta$ for $t$ in everything we just did.

But hold on a second!
The whole point of this was that $x = \cos \theta$ and $y = \sin \theta$.
Thus, we've really just proved the following:

\begin{theorem}
  \label{thm:deriv.sincos}
  The derivatives of the sine and cosine functions are given by
  \begin{align}
    \D{}{\theta} \sin \theta &= \cos \theta \label{eq:deriv.sin} \\
    \D{}{\theta} \cos \theta &= -\sin \theta. \label{eq:deriv.cos}
  \end{align}
  where $\theta$ is measured in radians.
\end{theorem}

It is extremely important to remember that \emph{\cref{thm:deriv.sincos} is only true if $\theta$ is measured in radians!}
If $\theta$ is instead measured in degrees or some other unit, different, less-satisfying versions of \cref{eq:deriv.sin,eq:deriv.cos} are obtained.
We'll revisit this issue in \cref{exc:deriv.trig.degrees}.

We could use similar methods to find the derivatives of the remaining four trig functions as well!
However, we'll leave this aside for now; they'll be almost trivial to find once we develop a few more tools, so we'll do it in \cref{sec:deriv.trig.other}.

We also could have proved \cref{eq:deriv.sin,eq:deriv.cos} analytically using \cref{def:deriv.func}.
This approach is less geometrically enlightening, but it's a worthwhile exercise---so we've left it as \cref{exc:deriv.sin,exc:deriv.cos}.

Before we leave this behind, though, we should take a look at how the higher derivatives of $\sin$ and $\cos$ work out.
We've just shown that $\D{}{x} \sin x = \cos x$ and that $\D{}{x} \cos x = -\sin x$, so clearly $\D[2]{}{x} \sin x = -\sin x$.
Similarly, $\D[2]{}{x} \cos x = -\cos x$.
Thus, the sine and cosine functions are solutions of the \emph{differential equation}
\begin{equation*}
  f''(x) = -f(x).
\end{equation*}
This will make them extremely important in the study of a wide variety of physical phenomena, as we'll see in \cref{sec:differential-equations}.

\begin{exercises}
  \begin{exc}
    \label{exc:deriv.sin}
    Prove \cref{eq:deriv.sin} using \cref{def:deriv.func}.
    You may use the fact that $\lim_{x \to 0} \frac{\sin x}{x} = 1$ and the \enquote{angle sum formula}
    \begin{equation*}
      \sin \pbrac*{a + b} = \sin a \cos b + \cos a \sin b.
    \end{equation*}
  \end{exc}

  \begin{exc}
    \label{exc:deriv.cos}
    Prove \cref{eq:deriv.cos} using \cref{def:deriv.func}.
    You may use the fact that $\lim_{x \to 0} \frac{\cos x - 1}{x} = 0$ and the \enquote{angle sum formula}
    \begin{equation*}
      \cos \pbrac*{a + b} = \cos a \cos b + \sin a \sin b.
    \end{equation*}
  \end{exc}
\end{exercises}
\end{document}

%%% Local Variables:
%%% mode: latex
%%% TeX-master: "../book/calcnotes.tex"
%%% End: